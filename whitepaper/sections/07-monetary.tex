% ============================================================================
% Section 7: Monetary Policy
% ============================================================================

\section{Monetary Policy}
\label{sec:monetary}

\Botho's monetary policy balances long-term security sustainability with
controlled inflation, using dynamic mechanisms that respond to network
conditions.

\subsection{Design Goals}

\begin{enumerate}
  \item \textbf{Sustainable security}: Perpetual miner incentive without
    relying solely on transaction fees.
  \item \textbf{Fair distribution}: Rewards distributed over time, not
    front-loaded to early adopters.
  \item \textbf{Anti-hoarding}: Economic pressure toward circulation rather
    than accumulation.
  \item \textbf{Predictability}: Clear, auditable emission schedule.
\end{enumerate}

\subsection{Emission Schedule}

\subsubsection{Initial Emission}

Block rewards follow a smooth decay curve rather than Bitcoin's abrupt
halvings:

\begin{equation}
R(h) = R_0 \cdot \left(\frac{1}{2}\right)^{h/H}
\end{equation}

where:
\begin{itemize}
  \item $R(h)$ is the block reward at height $h$
  \item $R_0 = 50$ BTH is the initial block reward
  \item $H = 1{,}051{,}200$ blocks ($\approx$ 2 years at 60s blocks)
    is the halving period
\end{itemize}

This creates continuous, gradual reduction rather than discrete shocks.

\subsubsection{Tail Emission}

Once block rewards decay below a threshold, perpetual tail emission begins:

\begin{equation}
R_{\text{tail}} = \max\left(R(h), 0.3 \text{ BTH}\right)
\end{equation}

At 5-second blocks (12 blocks per minute), this produces:
\begin{equation}
\text{Annual tail emission} = 0.3 \times 12 \times 60 \times 24 \times 365 = 1{,}892{,}160 \text{ BTH}
\end{equation}

\subsubsection{Supply Projection}

\begin{table}[h]
\centering
\caption{Supply projection}
\label{tab:supply}
\begin{tabular}{@{}rrrr@{}}
\toprule
\textbf{Year} & \textbf{Circulating} & \textbf{New Emission} & \textbf{Inflation} \\
\midrule
1 & 15.8M BTH & 15.8M & --- \\
2 & 23.7M BTH & 7.9M & 50\% \\
3 & 27.6M BTH & 3.9M & 17\% \\
5 & 30.2M BTH & 1.0M & 3.4\% \\
10 & 33.1M BTH & 1.9M & 6.1\% \\
20 & 52.0M BTH & 1.9M & 3.8\% \\
\bottomrule
\end{tabular}
\end{table}

\textbf{Note}: Tail emission inflation is asymptotically declining---while
nominal emission is constant, the percentage decreases as supply grows.
After 50 years, annual inflation is approximately 2.1\%.

\subsection{Dynamic Block Timing}
\label{sec:dynamic-timing}

\Botho introduces adaptive block intervals that respond to network
utilization, creating natural inflation dampening.

\subsubsection{Mechanism}

Block time targets range from 5 to 40 seconds based on mempool pressure:

\begin{equation}
T_{\text{target}} = T_{\min} + (T_{\max} - T_{\min}) \cdot (1 - U)
\end{equation}

where:
\begin{itemize}
  \item $T_{\min} = 5$ seconds (minimum block time)
  \item $T_{\max} = 40$ seconds (maximum block time)
  \item $U \in [0, 1]$ is the normalized utilization factor
\end{itemize}

\subsubsection{Utilization Calculation}

Utilization is computed from recent mempool activity:

\begin{equation}
U = \text{sigmoid}\left(\frac{\text{mempool\_weight} - \text{target\_weight}}{\text{sensitivity}}\right)
\end{equation}

\begin{table}[h]
\centering
\caption{Block timing by utilization}
\label{tab:timing-util}
\begin{tabular}{@{}lcc@{}}
\toprule
\textbf{Network State} & \textbf{Utilization} & \textbf{Block Time} \\
\midrule
Idle (no transactions) & 0\% & 40s \\
Light usage & 25\% & 31s \\
Moderate usage & 50\% & 23s \\
Heavy usage & 75\% & 14s \\
Maximum demand & 100\% & 5s \\
\bottomrule
\end{tabular}
\end{table}

\subsubsection{Economic Implications}

Dynamic timing creates feedback loops:

\textbf{Low utilization}:
\begin{itemize}
  \item Longer block times $\rightarrow$ fewer blocks per hour
  \item Fewer blocks $\rightarrow$ less emission
  \item Less emission $\rightarrow$ reduced inflation when currency is
    unused
\end{itemize}

\textbf{High utilization}:
\begin{itemize}
  \item Shorter block times $\rightarrow$ more blocks per hour
  \item More blocks $\rightarrow$ higher throughput
  \item Higher throughput $\rightarrow$ more rewards when currency is
    actively used
\end{itemize}

This aligns miner incentives with network utility.

\subsubsection{Emission Bounds}

The effective annual emission varies with utilization:

\begin{table}[h]
\centering
\caption{Tail emission by utilization}
\label{tab:emission-util}
\begin{tabular}{@{}lcc@{}}
\toprule
\textbf{Utilization} & \textbf{Blocks/Year} & \textbf{Emission/Year} \\
\midrule
0\% (idle) & 788,400 & 236,520 BTH \\
50\% (moderate) & 1,370,087 & 411,026 BTH \\
100\% (maximum) & 6,307,200 & 1,892,160 BTH \\
\bottomrule
\end{tabular}
\end{table}

\subsection{Fee Economics}

\subsubsection{Base Fee}

The minimum fee per byte is:

\begin{equation}
f_{\text{base}} = 1 \text{ nano-BTH per byte}
\end{equation}

For a typical 4 KB transaction:
\begin{equation}
f_{\text{min}} = 4{,}000 \times 1 = 4{,}000 \text{ nano-BTH} = 0.000004 \text{ BTH}
\end{equation}

\subsubsection{Progressive Fee Multiplier}

Transactions pay additional fees based on cluster wealth (see
Section~\ref{sec:cluster-tags}):

\begin{equation}
f_{\text{total}} = f_{\text{base}} \times \text{size} \times \text{cluster\_factor}
\end{equation}

The cluster factor ranges from 1.0$\times$ (bottom 50\% wealth) to
6.0$\times$ (top 1\% wealth).

\subsubsection{Lottery-Based Fee Redistribution}

Transaction fees are split between redistribution and burning:

\begin{itemize}
  \item \textbf{80\% redistributed}: Immediately distributed to 4 randomly
    selected UTXOs via verifiable lottery
  \item \textbf{20\% burned}: Permanently removed from supply
\end{itemize}

This creates:

\begin{itemize}
  \item \textbf{Progressive redistribution}: Random UTXO selection statistically
    favors the many (small holders) over the few (large holders)
  \item \textbf{Anti-custodial incentive}: Exchanges holding user funds in few
    UTXOs receive less redistribution than self-custody users with many UTXOs
  \item \textbf{Deflationary pressure}: 20\% burn creates supply reduction
  \item \textbf{Anti-MEV}: No miner incentive to reorder transactions
  \item \textbf{Alignment}: Miner income comes from block rewards only
\end{itemize}

\subsubsection{Lottery Mechanism}

For each transaction:

\begin{enumerate}
  \item Calculate fee $f$ based on cluster factor and transaction size
  \item Burn $0.2f$ (20\%)
  \item Select 4 random UTXOs weighted by eligibility criteria
  \item Distribute $0.2f$ to each selected UTXO (80\% total)
\end{enumerate}

The lottery uses the previous block hash as verifiable randomness, ensuring
selection cannot be manipulated by transaction creators.

\subsubsection{Effective Inflation}

The effective inflation rate becomes:

\begin{equation}
\text{inflation}_{\text{effective}} = \frac{\text{emission} - \text{fees\_burned}}{\text{supply}}
\end{equation}

where $\text{fees\_burned} = 0.2 \times \text{total\_fees}$. Under high
utilization with progressive fees, burning may exceed tail emission,
creating net deflation.

\subsubsection{Fee Flow Analysis}

\begin{table}[h]
\centering
\caption{Fee flow scenarios (1M daily transactions)}
\label{tab:fee-flow}
\begin{tabular}{@{}lrrr@{}}
\toprule
\textbf{Wealth Distribution} & \textbf{Daily Fees} & \textbf{Redistributed} & \textbf{Burned} \\
\midrule
Equal (factor 1.0) & 16 BTH & 12.8 BTH & 3.2 BTH \\
Current inequality & 48 BTH & 38.4 BTH & 9.6 BTH \\
High inequality & 96 BTH & 76.8 BTH & 19.2 BTH \\
\bottomrule
\end{tabular}
\end{table}

\subsection{Minter Incentives}

\subsubsection{Block Reward Composition}

Minters receive only the block reward:

\begin{equation}
\text{minter\_income} = R(h) \text{ (no fees)}
\end{equation}

This simplifies minting economics and eliminates fee-based attacks.

\subsubsection{Profitability Analysis}

Minting profitability depends on:
\begin{itemize}
  \item Hardware cost (one-time)
  \item Electricity cost (ongoing)
  \item Block reward value
  \item Network hashrate (competition)
\end{itemize}

The tail emission ensures perpetual profitability for efficient miners,
unlike Bitcoin where fee-only income creates uncertainty.

\subsubsection{Decentralization Incentives}

Several mechanisms encourage mining decentralization:

\begin{itemize}
  \item \textbf{CPU-friendly PoW}: Algorithm resists ASIC development
    (RandomX-based)
  \item \textbf{Solo mining viable}: Tail emission ensures consistent
    income even at low hashrate
  \item \textbf{No economy of scale}: Linear scaling of rewards with
    hashpower
\end{itemize}

\subsection{Long-Term Sustainability}

\subsubsection{Security Budget}

The perpetual security budget (in BTH terms) is:

\begin{equation}
\text{security\_budget} = R_{\text{tail}} \times \text{blocks\_per\_year}
\end{equation}

This ranges from 236K BTH (idle) to 1.89M BTH (maximum utilization) annually.

\subsubsection{Comparison with Fixed Supply}

\begin{table}[h]
\centering
\caption{Security model comparison}
\label{tab:security-comparison}
\begin{tabular}{@{}lcc@{}}
\toprule
\textbf{Aspect} & \textbf{Fixed Supply} & \textbf{Tail Emission} \\
\midrule
Long-term miner income & Fees only & Rewards + fees \\
Security if fees low & Degraded & Maintained \\
Inflation & 0\% & $\sim$2--3\% \\
Predictability & High & High \\
\bottomrule
\end{tabular}
\end{table}

\subsubsection{Wealth Tax Equivalence}

Tail emission functions as a mild wealth tax:

\begin{equation}
\text{effective\_tax} = \frac{\text{tail\_emission}}{\text{total\_supply}} \approx 2\%
\end{equation}

This encourages circulation: holding idle currency means gradual dilution,
while active participation maintains relative wealth.

\subsection{Monetary Constants}

\begin{table}[h]
\centering
\caption{Monetary policy constants}
\label{tab:constants}
\begin{tabular}{@{}lrl@{}}
\toprule
\textbf{Parameter} & \textbf{Value} & \textbf{Description} \\
\midrule
Initial reward & 50 BTH & Block reward at genesis \\
Halving period & 1,051,200 blocks & $\approx$ 2 years \\
Tail emission & 0.3 BTH & Minimum block reward \\
Min block time & 5 seconds & At maximum utilization \\
Max block time & 40 seconds & At zero utilization \\
Base fee rate & 1 nano-BTH/byte & Minimum transaction fee \\
Max cluster factor & 6.0$\times$ & Top 1\% wealth multiplier \\
Decimals & 9 & Smallest unit: 1 nano-BTH \\
\bottomrule
\end{tabular}
\end{table}

