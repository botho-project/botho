% ============================================================================
% Section 3: Preliminaries
% ============================================================================

\section{Preliminaries}
\label{sec:preliminaries}

\subsection{Notation}

We establish the following notation used throughout this paper:

\begin{center}
\begin{tabular}{cl}
\toprule
\textbf{Symbol} & \textbf{Description} \\
\midrule
$\lambda$ & Security parameter \\
$\G$ & Elliptic curve group (Ristretto255) \\
$G$ & Generator of $\G$ \\
$H$ & Secondary generator for Pedersen commitments \\
$q$ & Order of $\G$ (prime) \\
$\Zq$ & Integers modulo $q$ \\
$\Rq$ & Polynomial ring $\mathbb{Z}_q[X]/(X^n + 1)$ for lattice operations \\
\midrule
$a, b$ & View and spend private keys (scalars) \\
$A, B$ & View and spend public keys (points) \\
$(C, D)$ & Subaddress public key pair \\
$P$ & One-time public key \\
$x$ & One-time private key \\
$I$ & Key image \\
\midrule
$\Hs(\cdot)$ & Hash function to scalar \\
$\Hp(\cdot)$ & Hash function to curve point \\
$\Hash(\cdot)$ & General cryptographic hash \\
\midrule
$\{P_i\}_{i=0}^{n-1}$ & Ring of public keys \\
$\pi$ & Secret index of real key in ring \\
$\sigma$ & Signature \\
$C_v$ & Pedersen commitment to value $v$ \\
\midrule
$\xleftarrow{\$}$ & Uniform random sampling \\
$\|$ & Concatenation \\
$\negl(\lambda)$ & Negligible function in $\lambda$ \\
\bottomrule
\end{tabular}
\end{center}

\subsection{Cryptographic Assumptions}

The security of \Botho relies on the following computational assumptions:

\begin{definition}[Discrete Logarithm Problem (DLP)]
Given $G \in \G$ and $Y = xG$ for random $x \xleftarrow{\$} \Zq$, compute $x$.
\end{definition}

\begin{definition}[Decisional Diffie-Hellman (DDH)]
Distinguish $(G, aG, bG, abG)$ from $(G, aG, bG, cG)$ for random
$a, b, c \xleftarrow{\$} \Zq$.
\end{definition}

\begin{definition}[Module Learning With Errors (MLWE)]
Given $(\mathbf{A}, \mathbf{A}\mathbf{s} + \mathbf{e})$ where
$\mathbf{A} \xleftarrow{\$} \Rq^{k \times l}$,
$\mathbf{s} \xleftarrow{\$} \Rq^l$, and
$\mathbf{e}$ is a small error vector,
distinguish from $(\mathbf{A}, \mathbf{u})$ where
$\mathbf{u} \xleftarrow{\$} \Rq^k$.
\end{definition}

\begin{definition}[Module Short Integer Solution (MSIS)]
Given $\mathbf{A} \xleftarrow{\$} \Rq^{k \times l}$, find
$\mathbf{x} \in \Rq^l$ with $\mathbf{Ax} = \mathbf{0}$ and
$\|\mathbf{x}\| \leq \beta$ for bound $\beta$.
\end{definition}

\subsection{Building Blocks}

\subsubsection{Elliptic Curve Group}

\Botho uses the Ristretto255 group~\cite{ristretto}, a prime-order group
constructed from Curve25519~\cite{curve25519}. Ristretto eliminates cofactor-related
complexities, providing a clean abstraction with:
\begin{itemize}
  \item Prime order $q = 2^{252} + 27742317777372353535851937790883648493$
  \item Efficient constant-time operations
  \item Resistance to small-subgroup attacks
\end{itemize}

\subsubsection{Pedersen Commitments}

A Pedersen commitment~\cite{pedersen} to value $v$ with blinding factor
$r$ is:
\begin{equation}
C = vH + rG
\end{equation}

where $G$ and $H$ are independent generators (the discrete log of $H$
with respect to $G$ is unknown).

\begin{theorem}[Pedersen Commitment Properties]
Pedersen commitments satisfy:
\begin{enumerate}
  \item \textbf{Hiding} (information-theoretic): $C$ reveals nothing
    about $v$ without $r$.
  \item \textbf{Binding} (computational): Finding $(v', r') \neq (v, r)$
    with the same commitment requires solving DLP.
  \item \textbf{Homomorphic}: $C_1 + C_2 = (v_1 + v_2)H + (r_1 + r_2)G$.
\end{enumerate}
\end{theorem}

The homomorphic property enables verification that transaction inputs
equal outputs plus fee, without revealing individual amounts.

\subsubsection{Bulletproofs}

Bulletproofs~\cite{bulletproofs} provide zero-knowledge range proofs with
logarithmic size. For a commitment $C = vH + rG$, a Bulletproof proves:
\begin{equation}
v \in [0, 2^{64})
\end{equation}

This prevents negative amounts (which would allow coin creation) while
keeping amounts confidential.

\textbf{Aggregation}: Multiple range proofs can be aggregated, with size
growing logarithmically in the number of proofs. A transaction with $m$
outputs has proof size $O(\log m)$.

\subsubsection{ML-KEM-768}

ML-KEM (Module-Lattice-based Key-Encapsulation Mechanism)~\cite{fips203,kyber},
formerly known as Kyber, is a NIST-standardized post-quantum key
encapsulation mechanism.

\begin{itemize}
  \item $(\pk, \sk) \leftarrow \MLKEM.\KeyGen()$: Generate keypair
  \item $(c, K) \leftarrow \MLKEM.\Encap(\pk)$: Encapsulate to shared secret
  \item $K \leftarrow \MLKEM.\Decap(\sk, c)$: Decapsulate to recover secret
\end{itemize}

ML-KEM-768 provides approximately 192-bit security against quantum
attacks, with:
\begin{itemize}
  \item Public key size: 1,184 bytes
  \item Ciphertext size: 1,088 bytes
  \item Shared secret size: 32 bytes
\end{itemize}

\subsubsection{ML-DSA-65}

ML-DSA (Module-Lattice-based Digital Signature Algorithm)~\cite{fips204,dilithium},
formerly known as Dilithium, is a NIST-standardized post-quantum signature
scheme.

\begin{itemize}
  \item $(\pk, \sk) \leftarrow \MLDSA.\KeyGen()$: Generate keypair
  \item $\sigma \leftarrow \MLDSA.\Sign(\sk, m)$: Sign message
  \item $\{0,1\} \leftarrow \MLDSA.\Verify(\pk, m, \sigma)$: Verify signature
\end{itemize}

ML-DSA-65 (security level 3) provides approximately 128-bit security
against quantum attacks, with:
\begin{itemize}
  \item Public key size: 1,952 bytes
  \item Secret key size: 4,032 bytes
  \item Signature size: 3,309 bytes
\end{itemize}

\subsection{Security Definitions}

\begin{definition}[Unforgeability (EUF-CMA)]
A signature scheme is \textit{existentially unforgeable under chosen
message attack} if no PPT adversary, given access to a signing oracle,
can produce a valid signature on a message not queried to the oracle,
except with negligible probability.
\end{definition}

\begin{definition}[Anonymity (Ring Signatures)]
A ring signature scheme provides \textit{anonymity} if no PPT adversary
can identify the actual signer among ring members with probability
significantly better than $1/n$, where $n$ is the ring size.
\end{definition}

\begin{definition}[Linkability]
A ring signature scheme is \textit{linkable} if there exists an efficient
algorithm that, given two signatures from the same secret key, outputs 1;
and given signatures from different keys, outputs 0 (with high probability).
\end{definition}

\begin{definition}[IND-CCA2 Security]
A key encapsulation mechanism is \textit{IND-CCA2 secure} if no PPT
adversary with access to a decapsulation oracle can distinguish real
from random shared secrets, except with negligible advantage.
\end{definition}
