% ============================================================================
% Section 4: Cryptographic Protocol
% ============================================================================

\section{Cryptographic Protocol}
\label{sec:cryptography}

\subsection{Key Hierarchy}

\Botho derives all cryptographic keys from a single BIP39~\cite{bip39}
mnemonic using SLIP-10~\cite{slip10} hierarchical derivation.

\subsubsection{Master Seed Generation}

A 24-word mnemonic encodes 256 bits of entropy. The master seed is derived
via:
\begin{equation}
\text{seed} = \text{PBKDF2-SHA512}(\text{mnemonic}, \text{``mnemonic''} \| \text{passphrase}, 2048)
\end{equation}

This produces a 512-bit seed from which all keys are derived.

\subsubsection{Account Key Derivation}

Using SLIP-10 hardened derivation with path $m/44'/866'/\text{account}'$:

\begin{align}
\text{view\_key\_material} &= \text{SLIP10}(\text{seed}, m/44'/866'/0'/0') \\
\text{spend\_key\_material} &= \text{SLIP10}(\text{seed}, m/44'/866'/0'/1')
\end{align}

These materials are then processed through HKDF-SHA512~\cite{rfc5869} to produce the
final Ristretto255 scalars:

\begin{align}
a &= \text{HKDF}(\text{view\_key\_material}, \text{``botho-ristretto255-view''}) \mod q \\
b &= \text{HKDF}(\text{spend\_key\_material}, \text{``botho-ristretto255-spend''}) \mod q
\end{align}

The corresponding public keys are:
\begin{equation}
A = aG, \quad B = bG
\end{equation}

The public address is the tuple $(A, B)$.

% Key Derivation Hierarchy Diagram
% Shows the flow from mnemonic to all derived keys
%
% ACCESSIBILITY ALT TEXT:
% A flowchart showing cryptographic key derivation in Botho. Starting from a
% 24-word mnemonic phrase at the top, arrows flow downward through BIP39 to
% a 512-bit seed, then SLIP-10 derivation produces classical keys (view key a,
% spend key b) shown in blue boxes on the left, and post-quantum keys (ML-KEM,
% ML-DSA) shown in green boxes on the right. The classical keys derive public
% keys A and B, which combine to form subaddresses (C_i, D_i) at the bottom.

\begin{figure}[ht]
\centering
\begin{tikzpicture}[
    node distance=1.2cm and 2cm,
    box/.style={rectangle, draw, rounded corners, minimum width=2.5cm, minimum height=0.8cm, align=center, font=\small},
    keybox/.style={rectangle, draw, fill=blue!10, rounded corners, minimum width=2.2cm, minimum height=0.7cm, align=center, font=\small},
    pqbox/.style={rectangle, draw, fill=green!10, rounded corners, minimum width=2.2cm, minimum height=0.7cm, align=center, font=\small},
    arrow/.style={->, >=stealth, thick},
    label/.style={font=\scriptsize, midway, above},
]

% Top level - Mnemonic
\node[box, fill=yellow!20] (mnemonic) {24-word Mnemonic\\(256 bits entropy)};

% Seed
\node[box, fill=orange!15, below=of mnemonic] (seed) {Master Seed\\(PBKDF2)};

% SLIP-10 derivation branches
\node[keybox, below left=1.5cm and 1cm of seed] (viewmat) {View Key Material\\m/44'/866'/0'/0'};
\node[keybox, below right=1.5cm and 1cm of seed] (spendmat) {Spend Key Material\\m/44'/866'/0'/1'};

% HKDF derived keys
\node[keybox, below=of viewmat] (viewkey) {View Private Key\\$a \in \mathbb{Z}_q$};
\node[keybox, below=of spendmat] (spendkey) {Spend Private Key\\$b \in \mathbb{Z}_q$};

% Public keys
\node[keybox, below=of viewkey] (viewpub) {View Public Key\\$A = aG$};
\node[keybox, below=of spendkey] (spendpub) {Spend Public Key\\$B = bG$};

% PQ Keys (derived separately)
\node[pqbox, right=2.5cm of spendmat] (pqkem) {ML-KEM Keypair\\(Post-Quantum)};
\node[pqbox, below=of pqkem] (pqdsa) {ML-DSA Keypair\\(Post-Quantum)};

% Subaddresses
\node[box, fill=purple!10, below=1.5cm of $(viewpub)!0.5!(spendpub)$] (subaddr) {Subaddresses\\$(C_i, D_i)$ for index $i$};

% Arrows
\draw[arrow] (mnemonic) -- (seed) node[label] {PBKDF2};
\draw[arrow] (seed) -- (viewmat) node[label, sloped] {SLIP-10};
\draw[arrow] (seed) -- (spendmat) node[label, sloped] {SLIP-10};
\draw[arrow] (viewmat) -- (viewkey) node[label] {HKDF};
\draw[arrow] (spendmat) -- (spendkey) node[label] {HKDF};
\draw[arrow] (viewkey) -- (viewpub) node[label] {$\times G$};
\draw[arrow] (spendkey) -- (spendpub) node[label] {$\times G$};
\draw[arrow] (viewpub) -- (subaddr);
\draw[arrow] (spendpub) -- (subaddr);

% PQ derivation
\draw[arrow] (seed) -- ++(3.5,0) |- (pqkem) node[pos=0.25, above, font=\scriptsize] {DeriveKEM};
\draw[arrow] (pqkem) -- (pqdsa) node[label] {DeriveDSA};

% Grouping box for classical keys
\begin{scope}[on background layer]
\node[draw, dashed, rounded corners, fit=(viewkey)(spendkey)(viewpub)(spendpub),
      label={[font=\scriptsize]above:Classical (Ristretto255)}] {};
\end{scope}

\end{tikzpicture}
\caption{Key derivation hierarchy. A single BIP39 mnemonic derives all keys via
SLIP-10 paths and HKDF domain separation. Classical Ristretto255 keys support
ring signatures, while post-quantum ML-KEM/ML-DSA keys protect recipient privacy
and minting operations.}
\label{fig:key-hierarchy}
\end{figure}


\subsubsection{Subaddress Derivation}

Subaddresses allow generation of unlimited unlinkable addresses from a
single master key pair. For subaddress index $i$:

\begin{align}
\delta_i &= \Hs(\text{``SubAddr''} \| a \| i) \\
c_i &= a + \delta_i \mod q \\
d_i &= b + \delta_i \mod q
\end{align}

The subaddress public key pair is:
\begin{equation}
(C_i, D_i) = (c_i G, d_i G) = (A + \delta_i G, B + \delta_i G)
\end{equation}

\begin{theorem}[Subaddress Unlinkability]
Without knowledge of the view key $a$, subaddresses $(C_i, D_i)$ and
$(C_j, D_j)$ for $i \neq j$ are computationally indistinguishable from
independent random points.
\end{theorem}

\subsection{Post-Quantum Stealth Addresses}
\label{sec:pq-stealth}

Traditional CryptoNote stealth addresses use ECDH for key exchange. \Botho
replaces this with ML-KEM-768 to achieve post-quantum recipient privacy.

\subsubsection{Protocol Description}

Let the recipient have address $(A, B)$ where $A$ is interpreted as an
ML-KEM public key (derived via a domain-separated hash from the Ristretto
point).

\textbf{Sender} (creating output for recipient):
\begin{enumerate}
  \item Derive ML-KEM public key: $\pk_{\text{kem}} = \text{DeriveKEM}(A)$
  \item Encapsulate: $(c, K) \leftarrow \MLKEM.\Encap(\pk_{\text{kem}})$
  \item Compute scalar: $s = \Hs(K \| \text{output\_index})$
  \item Compute one-time public key: $P = sG + B$
  \item Include ciphertext $c$ (1,088 bytes) in transaction output
\end{enumerate}

\textbf{Recipient} (scanning for received outputs):
\begin{enumerate}
  \item Derive ML-KEM secret key: $\sk_{\text{kem}} = \text{DeriveKEM}(a)$
  \item For each output with ciphertext $c$:
    \begin{enumerate}
      \item Decapsulate: $K \leftarrow \MLKEM.\Decap(\sk_{\text{kem}}, c)$
      \item Compute scalar: $s' = \Hs(K \| \text{output\_index})$
      \item Compute expected key: $P' = s'G + B$
      \item If $P' = P$ (output's public key), this output belongs to us
      \item Compute private key: $x = s' + b \mod q$
    \end{enumerate}
\end{enumerate}

\textbf{Verification}: The one-time private key $x$ satisfies
$xG = (s + b)G = sG + B = P$.

\subsubsection{Security Analysis}

\begin{theorem}[Recipient Unlinkability]
Under the IND-CCA2 security of ML-KEM-768, an adversary (including a
quantum adversary) cannot link outputs to recipients with probability
better than negligible, given only the blockchain.
\end{theorem}

\begin{proof}
We prove by reduction to ML-KEM IND-CCA2 security. Suppose adversary
$\Adv$ can distinguish outputs belonging to recipient $R$ with
non-negligible advantage $\epsilon$. We construct adversary $\Adv'$
that breaks ML-KEM IND-CCA2 with advantage $\epsilon/2$.

\textbf{Setup}: $\Adv'$ receives ML-KEM public key $\pk^*$ from the
IND-CCA2 challenger. $\Adv'$ sets this as the target recipient's
KEM public key.

\textbf{Query phase}: $\Adv$ may request outputs for various recipients.
For non-target recipients, $\Adv'$ generates honestly. For the target
recipient, $\Adv'$ uses the decapsulation oracle.

\textbf{Challenge}: $\Adv$ submits two recipients $R_0, R_1$ (one being
the target). $\Adv'$ queries the IND-CCA2 challenger with
$(\pk^*, \pk_{\text{other}})$ and receives ciphertext $c^*$ encapsulating
either $K_0$ or $K_1$ (random bit $b$).

$\Adv'$ computes $s^* = \Hs(K_b \| \text{index})$ and $P^* = s^*G + B_b$,
returning output $(c^*, P^*)$ to $\Adv$.

\textbf{Analysis}: If $\Adv$ correctly identifies the recipient with
probability $1/2 + \epsilon$, then $\Adv'$ correctly guesses $b$ with
the same probability, contradicting IND-CCA2 security.

The reduction is tight: $\text{Adv}^{\text{IND-CCA2}}_{\Adv'} \geq \epsilon/2$.
\end{proof}

\begin{corollary}[Quantum Security]
The above reduction holds against quantum adversaries since ML-KEM-768
provides IND-CCA2 security in the quantum random oracle model (QROM).
\end{corollary}

% Post-Quantum Stealth Address Flow Diagram
% Shows sender encapsulation and recipient decapsulation
%
% ACCESSIBILITY ALT TEXT:
% A sequence diagram showing private transaction flow between sender and
% recipient. The sender (left) uses ML-KEM encapsulation to create a shared
% secret K, then derives a one-time public key P. The middle shows on-chain
% data: the KEM ciphertext c and stealth output P. The recipient (right)
% uses ML-KEM decapsulation to recover K, then derives the private key x to
% spend the funds. Green indicates post-quantum protected elements.

\begin{figure}[ht]
\centering
\begin{tikzpicture}[
    node distance=0.8cm and 1.5cm,
    box/.style={rectangle, draw, rounded corners, minimum width=2.2cm, minimum height=0.7cm, align=center, font=\small},
    pqbox/.style={box, fill=green!15},
    classbox/.style={box, fill=blue!15},
    databox/.style={box, fill=yellow!15},
    arrow/.style={->, >=stealth, thick},
    dasharrow/.style={->, >=stealth, dashed},
]

% === SENDER SIDE ===
\node[font=\bfseries] (sendlabel) {Sender};

\node[pqbox, below=0.5cm of sendlabel] (recvpk) {Recipient's\\ML-KEM PK};
\node[pqbox, below=of recvpk] (encap) {ML-KEM\\Encapsulate};
\node[databox, below=of encap] (shared) {Shared Secret $K$};
\node[classbox, below=of shared] (derive) {Derive $s = H_s(K \| i)$};
\node[classbox, below=of derive] (onetime) {One-time Key\\$P = sG + B$};

% Ciphertext output
\node[databox, right=2cm of encap] (cipher) {Ciphertext $c$\\(1,088 bytes)};

% === ON-CHAIN DATA ===
\node[font=\bfseries, right=3cm of sendlabel] (chainlabel) {On-Chain};
\node[box, fill=orange!15, below=0.5cm of chainlabel] (output) {Transaction\\Output};

% === RECIPIENT SIDE ===
\node[font=\bfseries, right=3cm of chainlabel] (recvlabel) {Recipient};
\node[pqbox, below=0.5cm of recvlabel] (recvsk) {ML-KEM SK\\(private)};
\node[pqbox, below=of recvsk] (decap) {ML-KEM\\Decapsulate};
\node[databox, below=of decap] (shared2) {Shared Secret $K$};
\node[classbox, below=of shared2] (derive2) {Derive $s = H_s(K \| i)$};
\node[classbox, below=of derive2] (privkey) {Spend Key\\$x = s + b$};

% Arrows - Sender flow
\draw[arrow] (recvpk) -- (encap);
\draw[arrow] (encap) -- (shared);
\draw[arrow] (shared) -- (derive);
\draw[arrow] (derive) -- (onetime);
\draw[arrow] (encap) -- (cipher);

% To chain
\draw[arrow] (cipher) -- (output);
\draw[arrow] (onetime) -| (output);

% From chain to recipient
\draw[arrow] (output) -| (decap);
\draw[arrow] (recvsk) -- (decap);
\draw[arrow] (decap) -- (shared2);
\draw[arrow] (shared2) -- (derive2);
\draw[arrow] (derive2) -- (privkey);

% Security annotations
\node[right=0.3cm of cipher, align=left, font=\scriptsize\color{green!50!black}] {PQ-secure\\(ML-KEM)};
\node[left=0.3cm of onetime, align=right, font=\scriptsize\color{blue!50!black}] {Classical\\(Ristretto)};

% Bottom annotation
\node[below=0.5cm of $(onetime)!0.5!(privkey)$, align=center, font=\footnotesize\itshape]
{Recipient identity protected by post-quantum cryptography;\\
only recipient can compute spend key $x$};

\end{tikzpicture}
\caption{Post-quantum stealth address protocol. The sender encapsulates to the
recipient's ML-KEM public key, deriving a shared secret used to create a unique
one-time public key $P$. The on-chain ciphertext $c$ is quantum-resistant---only
the recipient with the ML-KEM secret key can decapsulate and derive the
corresponding private key $x$ to spend the output.}
\label{fig:stealth-address}
\end{figure}


\subsection{Ring Signatures (CLSAG)}
\label{sec:clsag}

\CLSAG (Concise Linkable Spontaneous Anonymous Group)~\cite{clsag}
provides efficient linkable ring signatures~\cite{fujisaki2007} for sender privacy.

\subsubsection{Ring Construction}

For each input being spent, the sender:
\begin{enumerate}
  \item Selects $n-1$ decoy outputs from the blockchain (we use $n = 20$)
  \item Forms a ring $\mathcal{R} = \{(P_0, C_0), \ldots, (P_{n-1}, C_{n-1})\}$
    where each $(P_i, C_i)$ is a one-time public key and commitment
  \item The real input is at secret index $\pi$
\end{enumerate}

\subsubsection{Signature Generation}

Given:
\begin{itemize}
  \item Ring $\mathcal{R}$ with real index $\pi$
  \item Private key $x_\pi$ and commitment blinding factor $z_\pi$
  \item Message $m$ (transaction hash)
\end{itemize}

\textbf{Key Image}: $I = x_\pi \cdot \Hp(P_\pi)$

The key image is deterministic given the private key and serves as a
unique tag preventing double-spending.

\textbf{Aggregation Coefficients}:
\begin{align}
\mu_P &= \Hash(\text{``CLSAG\_agg\_0''} \| \mathcal{R} \| I \| D) \\
\mu_C &= \Hash(\text{``CLSAG\_agg\_1''} \| \mathcal{R} \| I \| D)
\end{align}

where $D$ is an auxiliary point for commitment verification.

\textbf{Signature}: The signature $\sigma = (c_0, s_0, \ldots, s_{n-1}, I, D)$
is computed via a Fiat-Shamir transform~\cite{fiatshamir} of an interactive protocol,
forming a ``ring'' of challenges that closes only if the prover knows
one of the private keys.

\subsubsection{Verification}

Given signature $\sigma$ and ring $\mathcal{R}$:
\begin{enumerate}
  \item Recompute aggregation coefficients $\mu_P$, $\mu_C$
  \item For $i = 0, \ldots, n-1$:
    \begin{align}
      W_i &= \mu_P P_i + \mu_C (C_i - C_{\text{out}}) \\
      L_i &= s_i G + c_i W_i \\
      R_i &= s_i \Hp(P_i) + c_i (\mu_P I + \mu_C D) \\
      c_{i+1} &= \Hash(\text{``CLSAG\_round''} \| \mathcal{R} \| L_i \| R_i \| m)
    \end{align}
  \item Accept if $c_n = c_0$ (ring closure)
\end{enumerate}

\subsubsection{Security Properties}

\begin{definition}[Unforgeability Game]
The unforgeability experiment $\text{Exp}^{\text{UNF}}_{\Adv}$ proceeds:
\begin{enumerate}
  \item Challenger generates key pairs $(x_i, P_i)$ for $i \in [n]$
  \item $\Adv$ receives $\{P_i\}$ and access to signing oracle
  \item $\Adv$ outputs $(m^*, \sigma^*, \mathcal{R}^*)$
  \item $\Adv$ wins if $\Verify(\sigma^*, m^*, \mathcal{R}^*) = 1$ and
    $m^*$ was never queried to signing oracle with ring $\mathcal{R}^*$
\end{enumerate}
\end{definition}

\begin{definition}[Anonymity Game]
The anonymity experiment $\text{Exp}^{\text{ANON}}_{\Adv}$ proceeds:
\begin{enumerate}
  \item Challenger generates key pairs, gives public keys to $\Adv$
  \item $\Adv$ selects ring $\mathcal{R}$, message $m$, two indices $i_0, i_1$
  \item Challenger flips bit $b$, signs with key $x_{i_b}$
  \item $\Adv$ outputs guess $b'$
  \item $\Adv$'s advantage: $|\Pr[b' = b] - 1/2|$
\end{enumerate}
\end{definition}

\begin{theorem}[CLSAG Security]
Under the DLP assumption in the random oracle model, CLSAG satisfies:
\begin{enumerate}
  \item \textbf{Unforgeability}: For all PPT $\Adv$,
    $\Pr[\text{Exp}^{\text{UNF}}_{\Adv} = 1] \leq \negl(\lambda)$.
  \item \textbf{Anonymity}: For all PPT $\Adv$,
    $\text{Adv}^{\text{ANON}}_{\Adv} \leq \negl(\lambda)$.
  \item \textbf{Linkability}: For all PPT $\Adv$,
    $\Pr[\text{same key, different images}] \leq \negl(\lambda)$ and
    $\Pr[\text{different keys, same image}] \leq \negl(\lambda)$.
\end{enumerate}
\end{theorem}

\begin{proof}[Proof (Unforgeability)]
By reduction to DLP. Given DLP instance $(G, Y = xG)$, embed $Y$ as one
ring member's public key. A forking lemma argument shows that a successful
forger can be rewound to extract the discrete log, contradicting DLP
hardness. Full proof in~\cite{clsag}.
\end{proof}

\begin{proof}[Proof (Anonymity)]
The signature is a Fiat-Shamir transform of a $\Sigma$-protocol. In the
random oracle model, the simulated transcript is indistinguishable from
real, regardless of which key was used. The ring structure ensures all
positions are computationally indistinguishable.
\end{proof}

\begin{proof}[Proof (Linkability)]
Key image $I = x \cdot \Hp(P)$ is deterministic given $(x, P)$. For the
same key, the same image is always produced. For different keys $x \neq x'$,
collision requires $x \cdot \Hp(P) = x' \cdot \Hp(P')$, which occurs with
probability $1/q$ (negligible) when $\Hp$ is modeled as a random oracle.
\end{proof}

% Ring Signature Concept Diagram
% Shows real input hidden among decoys
%
% ACCESSIBILITY ALT TEXT:
% A diagram showing how ring signatures hide the true signer. On the left,
% 20 circles represent possible signers (ring members), with one highlighted
% in red as the real signer and others in gray as decoys. Arrows from the
% real signer show key image derivation I = x * Hp(P). On the right, the
% CLSAG signature output shows that verifiers cannot determine which ring
% member actually signed, providing sender anonymity.

\begin{figure}[ht]
\centering
\begin{tikzpicture}[
    node distance=0.6cm,
    utxo/.style={circle, draw, minimum size=1cm, font=\small},
    realutxo/.style={utxo, fill=green!30, line width=2pt},
    decoy/.style={utxo, fill=gray!20},
    keyimg/.style={rectangle, draw, rounded corners, fill=red!20, minimum width=1.5cm, minimum height=0.6cm, font=\small},
    arrow/.style={->, >=stealth, thick},
]

% Ring of outputs (20 members)
\def\numring{12}  % Show 12 for visual clarity
\def\radius{3cm}

% Draw ring members
\foreach \i in {1,...,\numring} {
    \pgfmathsetmacro{\angle}{90 - (\i-1) * 360 / \numring}
    \ifnum\i=4
        % Real input (highlighted)
        \node[realutxo] (utxo\i) at (\angle:\radius) {$P_{\i}$};
    \else
        % Decoy
        \node[decoy] (utxo\i) at (\angle:\radius) {$P_{\i}$};
    \fi
}

% Center elements
\node[font=\bfseries] at (0, 0.5) {Ring};
\node[font=\small] at (0, 0) {$n = 20$};
\node[font=\scriptsize] at (0, -0.5) {(12 shown)};

% Key image
\node[keyimg, below=4cm of utxo1] (keyimg) {Key Image $I$};

% Signature output
\node[rectangle, draw, rounded corners, fill=blue!15, minimum width=3cm,
      below=0.8cm of keyimg, font=\small] (sig) {CLSAG Signature $\sigma$};

% Arrows from ring to key image
\foreach \i in {1,...,\numring} {
    \draw[arrow, gray!50] (utxo\i) -- (keyimg);
}

% Highlight real connection
\draw[arrow, green!60!black, line width=2pt] (utxo4) -- (keyimg)
    node[midway, right, font=\scriptsize] {Real};

% Annotation for real vs decoy
\node[anchor=west, align=left, font=\scriptsize] at (4, 2) {
    \textcolor{green!60!black}{\rule{0.5cm}{2pt}} Real input (secret)\\
    \textcolor{gray}{\rule{0.5cm}{1pt}} Decoys (public)
};

% Key image explanation
\node[anchor=west, align=left, font=\scriptsize] at (2.5, -3.5) {
    $I = x \cdot H_p(P)$\\
    Unique per output\\
    Prevents double-spend
};

% Properties
\node[anchor=west, align=left, font=\footnotesize] at (-5.5, -2) {
    \textbf{Properties:}\\
    $\bullet$ Verifier cannot identify real input\\
    $\bullet$ Key image links spends (no double-spend)\\
    $\bullet$ Signature proves knowledge of one key
};

\end{tikzpicture}
\caption{Ring signature structure. The real input (green) is hidden among 19
decoy outputs selected from the blockchain. The signature proves the signer
knows one of the private keys without revealing which. The key image $I$ is
deterministically derived from the real input, enabling double-spend detection
while preserving anonymity.}
\label{fig:ring-signature}
\end{figure}


\subsection{Confidential Transactions}

% Confidential Transaction Structure Diagram
% Shows Pedersen commitments, value conservation, and range proofs
%
% ACCESSIBILITY ALT TEXT:
% A diagram showing how transaction amounts are hidden using Pedersen
% commitments. Input commitments C_in (left boxes) contain hidden values
% v with blinding factors r. Output commitments C_out (right boxes)
% contain new hidden values. The equation shows that sum of inputs equals
% sum of outputs plus fee, verified without revealing actual amounts.
% Range proofs (Bulletproofs) attached below prove all values are positive.

\begin{figure}[ht]
\centering
\begin{tikzpicture}[
    box/.style={draw, rectangle, minimum width=2cm, minimum height=0.8cm, font=\scriptsize},
    commit/.style={draw, rectangle, rounded corners, fill=blue!10, minimum width=1.5cm, minimum height=0.6cm, font=\scriptsize},
    proof/.style={draw, rectangle, rounded corners, fill=green!10, minimum width=1.5cm, minimum height=0.6cm, font=\scriptsize},
    arrow/.style={->, >=stealth, thick}
]
    % Title
    \node[font=\small\bfseries] at (0,4.5) {Value Conservation with Hidden Amounts};

    % Inputs section
    \node[font=\scriptsize\bfseries] at (-4,3.5) {Inputs};

    % Input 1
    \node[commit] (cin1) at (-4,2.5) {$C_{in}^{(1)}$};
    \node[font=\tiny, below=0.1cm of cin1] {$v_1 H + r_1 G$};

    % Input 2
    \node[commit] (cin2) at (-4,1.2) {$C_{in}^{(2)}$};
    \node[font=\tiny, below=0.1cm of cin2] {$v_2 H + r_2 G$};

    % Plus sign
    \node at (-4,1.85) {$+$};

    % Outputs section
    \node[font=\scriptsize\bfseries] at (4,3.5) {Outputs};

    % Output 1
    \node[commit] (cout1) at (4,2.5) {$C_{out}^{(1)}$};
    \node[font=\tiny, below=0.1cm of cout1] {$v'_1 H + r'_1 G$};

    % Output 2
    \node[commit] (cout2) at (4,1.2) {$C_{out}^{(2)}$};
    \node[font=\tiny, below=0.1cm of cout2] {$v'_2 H + r'_2 G$};

    % Plus sign
    \node at (4,1.85) {$+$};

    % Fee
    \node[draw, rectangle, fill=orange!15, minimum width=1.5cm, minimum height=0.6cm, font=\scriptsize] (fee) at (4,-0.1) {Fee: $f \cdot H$};
    \node at (4,0.55) {$+$};

    % Conservation equation box
    \node[draw, rectangle, rounded corners, fill=gray!10, minimum width=6cm, minimum height=1cm, font=\small] (eq) at (0,-1.5) {
        $\displaystyle\sum_i C_{in}^{(i)} = \sum_j C_{out}^{(j)} + f \cdot H$
    };

    % Arrows showing flow
    \draw[arrow, blue!60] (-2.5,2) -- (-1,2) -- (-1,-1) -- (-2.5,-1.5);
    \draw[arrow, blue!60] (2.5,1.5) -- (1,1.5) -- (1,-1) -- (2.5,-1.5);

    % Range proofs
    \node[font=\scriptsize\bfseries] at (0,-3) {Range Proofs (Bulletproofs)};

    \node[proof] (rp1) at (-2,-3.8) {$\pi_1$};
    \node[font=\tiny, below=0.1cm of rp1] {$v'_1 \in [0, 2^{64})$};

    \node[proof] (rp2) at (2,-3.8) {$\pi_2$};
    \node[font=\tiny, below=0.1cm of rp2] {$v'_2 \in [0, 2^{64})$};

    % Arrows from outputs to range proofs
    \draw[arrow, green!60, dashed] (cout1) -- (4,-2.5) -- (2,-2.5) -- (rp2);
    \draw[arrow, green!60, dashed] (cout2) -- (4,-3) -- (2,-3.2);

    % Key insight box
    \node[draw, rectangle, rounded corners, fill=yellow!10, minimum width=7cm, font=\tiny, align=center] at (0,-5.2) {
        \textbf{Key Properties:}\\
        Validators verify $\sum C_{in} = \sum C_{out} + fH$ without learning $v_i$\\
        Range proofs prevent negative amounts (no coin creation)
    };

    % Legend
    \node[commit, minimum width=0.8cm] at (-4.5,-5.2) {};
    \node[font=\tiny, right] at (-3.8,-5.2) {Pedersen Commitment};
    \node[proof, minimum width=0.8cm] at (1,-5.2) {};
    \node[font=\tiny, right] at (1.7,-5.2) {Range Proof};

\end{tikzpicture}
\caption{Confidential transaction structure. Input and output values are hidden
in Pedersen commitments ($C = vH + rG$). Value conservation is verified via
commitment arithmetic: the sum of input commitments must equal output commitments
plus the public fee. Bulletproof range proofs ensure all output amounts are
non-negative, preventing coin creation through overflow attacks.}
\label{fig:confidential-tx}
\end{figure}


\subsubsection{Amount Commitment}

Each output commits to its amount $v$ with blinding factor $r$:
\begin{equation}
C = vH + rG
\end{equation}

The blinding factor $r$ is derived deterministically from the shared
secret to enable recipient recovery.

\subsubsection{Value Conservation}

For a transaction with inputs $\{C_{\text{in}}^{(i)}\}$ and outputs
$\{C_{\text{out}}^{(j)}\}$, value conservation requires:
\begin{equation}
\sum_i C_{\text{in}}^{(i)} = \sum_j C_{\text{out}}^{(j)} + fH
\end{equation}

where $f$ is the transaction fee (public). This holds if and only if:
\begin{equation}
\sum_i v_{\text{in}}^{(i)} = \sum_j v_{\text{out}}^{(j)} + f
\end{equation}

Validators check commitment arithmetic without learning individual values.

\subsubsection{Range Proofs}

Each output includes a Bulletproof demonstrating:
\begin{equation}
v_{\text{out}}^{(j)} \in [0, 2^{64})
\end{equation}

This prevents:
\begin{itemize}
  \item Negative amounts (which would create coins from nothing)
  \item Overflow attacks (amounts wrapping around)
\end{itemize}

\subsection{Minting Signatures}

Block rewards (minting transactions) use ML-DSA-65 signatures since the
minter's identity is public and must be verifiable long-term.

\begin{equation}
\sigma_{\text{mint}} = \MLDSA.\Sign(\sk_{\text{minter}}, \text{block\_hash} \| \text{nonce})
\end{equation}

The minter's ML-DSA public key is derived from the same seed as their
Ristretto keys, enabling unified key management.

\subsection{Hybrid Architecture Rationale}

\begin{table}[h]
\centering
\caption{Cryptographic choices by data lifetime}
\label{tab:hybrid}
\begin{tabular}{@{}llll@{}}
\toprule
\textbf{Data} & \textbf{Lifetime} & \textbf{Algorithm} & \textbf{Rationale} \\
\midrule
Recipient identity & Permanent & ML-KEM-768 & On-chain forever; must be PQ \\
Sender identity & Ephemeral & CLSAG & Value degrades; efficiency wins \\
Amounts & Permanent & Pedersen & Information-theoretic hiding \\
Minting authority & Permanent & ML-DSA-65 & Verifiable long-term \\
\bottomrule
\end{tabular}
\end{table}

\textbf{Why not full post-quantum?}

Post-quantum ring signatures (e.g., lattice-based constructions) impose
approximately 50$\times$ size overhead. A CLSAG signature is $\sim$700
bytes per input; a comparable lattice ring signature would be
$\sim$35 KB. With multiple inputs, transactions would exceed 100 KB,
making desktop nodes impractical.

\textbf{Why is ephemeral sender privacy acceptable?}

The value of sender deanonymization degrades over time. Learning who
sent a transaction in 2025 from a 2045 perspective has minimal economic
relevance---the goods have been delivered, contracts fulfilled, and
economic context forgotten. In contrast, recipient identity remains
valuable (``who owns this address?'') indefinitely.

This asymmetry justifies asymmetric protection: permanent data gets
permanent (post-quantum) protection; ephemeral data gets efficient
(classical) protection.
