% ============================================================================
% Section 9: Security Analysis
% ============================================================================

\section{Security Analysis}
\label{sec:security}

This section analyzes \Botho's security properties, threat models, and
resistance to various attacks.

\subsection{Threat Model}

\subsubsection{Adversary Capabilities}

We consider adversaries with the following capabilities:

\begin{itemize}
  \item \textbf{Network}: Can observe, delay, and inject messages (but
    not break encryption)
  \item \textbf{Computational}: Polynomial-time bounded (or BQP for
    quantum adversaries)
  \item \textbf{Economic}: Can acquire significant hashpower or stake
  \item \textbf{Passive}: Can collect and analyze blockchain data
    indefinitely
\end{itemize}

\subsubsection{Security Goals}

\begin{enumerate}
  \item \textbf{Safety}: No double-spending; no unauthorized coin creation
  \item \textbf{Liveness}: Valid transactions eventually confirm
  \item \textbf{Privacy}: Transaction graph analysis does not reveal
    sender, recipient, or amount
  \item \textbf{Quantum resistance}: Long-term data remains protected
    against quantum attacks
\end{enumerate}

\subsection{Privacy Analysis}

\subsubsection{Recipient Unlinkability}

\begin{theorem}[Post-Quantum Recipient Privacy]
Under the IND-CCA2 security of ML-KEM-768, an adversary with access to
a quantum computer cannot link outputs to recipient addresses with
probability better than negligible.
\end{theorem}

\begin{proof}
Each output contains:
\begin{itemize}
  \item ML-KEM ciphertext $c$ encapsulating shared secret $K$
  \item One-time key $P = sG + B$ where $s = \Hs(K \| \text{index})$
\end{itemize}

Without the recipient's secret key, the shared secret $K$ is computationally
indistinguishable from random (IND-CCA2 security holds against quantum
adversaries for ML-KEM-768). Thus $s$ is random, and $P$ reveals nothing
about the recipient's public key $B$.
\end{proof}

\subsubsection{Sender Anonymity}

\begin{theorem}[Ring Signature Anonymity]
Under the DLP assumption in the random oracle model, the probability that
an adversary identifies the true signer among ring members is at most
$1/n + \negl(\lambda)$, where $n = 20$ is the ring size.
\end{theorem}

\textbf{Caveat}: This is classical security. A quantum adversary with access
to a sufficiently large quantum computer could potentially solve DLP and
identify signers. However:
\begin{itemize}
  \item Sender identity has ephemeral value (see Section~\ref{sec:cryptography})
  \item Ring signature data is not uniquely attributable even with private
    key recovery
  \item Retroactive deanonymization provides limited advantage
\end{itemize}

\subsubsection{Amount Confidentiality}

\begin{theorem}[Information-Theoretic Amount Hiding]
Transaction amounts are unconditionally hidden: no computational power
(including quantum computers) can determine amounts from Pedersen
commitments alone.
\end{theorem}

\begin{proof}
A Pedersen commitment $C = vH + rG$ with random blinding factor $r$ is
uniformly distributed over the curve, independent of $v$. For any
commitment $C$ and any amount $v'$, there exists $r'$ such that
$C = v'H + r'G$. Without the discrete log of $H$ base $G$, the commitment
is perfectly hiding.
\end{proof}

\subsubsection{Anonymity Set Analysis}

The effective anonymity set depends on decoy selection quality:

\begin{table}[h]
\centering
\caption{Anonymity degradation factors}
\label{tab:anonymity}
\begin{tabular}{@{}lcc@{}}
\toprule
\textbf{Attack} & \textbf{Reduction} & \textbf{Mitigation} \\
\midrule
Timing analysis & $\sim$2--3$\times$ & Age distribution matching \\
Output age heuristic & $\sim$1.5$\times$ & Recent output bias \\
Cluster analysis & $\sim$1.2$\times$ & Cluster similarity requirement \\
Repeated use & Cumulative & Transaction spacing \\
\midrule
\textbf{Effective ring size} & $\sim$5--8 & (from nominal 20) \\
\bottomrule
\end{tabular}
\end{table}

Even with degradation, the effective anonymity set provides meaningful
privacy protection for typical use cases.

\subsection{Consensus Security}

\subsubsection{Double-Spend Resistance}

\begin{theorem}[Double-Spend Prevention]
Under SCP's safety guarantees and assuming quorum intersection, no valid
double-spend can be confirmed.
\end{theorem}

\begin{proof}
Consider an attempt to double-spend output $O$ in transactions $T_1$ and
$T_2$:
\begin{enumerate}
  \item Both transactions contain the same key image $I$
  \item Key images are checked for uniqueness before block inclusion
  \item If $T_1$ is included in block $B_1$ at height $h$, then $I$ is
    added to the key image set
  \item Any block $B_2$ at height $h' > h$ containing $T_2$ would fail
    validation (duplicate key image)
  \item By SCP safety, no alternative block can be externalized at height
    $h$
\end{enumerate}
\end{proof}

\subsubsection{Byzantine Fault Tolerance}

The system tolerates Byzantine nodes within quorum thresholds:

\begin{theorem}[Byzantine Resilience]
If quorum intersection holds and each quorum can tolerate its failure
threshold, honest nodes agree on the same externalized values.
\end{theorem}

With the default tiered structure (3-of-4 infrastructure + 2-of-3 community),
the system tolerates:
\begin{itemize}
  \item 1 Byzantine infrastructure node, OR
  \item 1 Byzantine community node, OR
  \item Combinations below both thresholds
\end{itemize}

\subsubsection{51\% Attack Resistance}

Unlike pure PoW, \Botho is resistant to hashpower-majority attacks:

\begin{itemize}
  \item \textbf{Block proposal}: Majority hashpower can propose blocks
    more frequently
  \item \textbf{Finalization}: SCP requires quorum agreement regardless
    of hashpower
  \item \textbf{Result}: Attacker can flood proposals but cannot force
    finalization
\end{itemize}

An attacker would need to compromise both hashpower majority AND sufficient
quorum members.

\subsection{Cryptographic Security}

\subsubsection{Hash Function Security}

\Botho uses SHA3-256 and BLAKE3 for hashing:
\begin{itemize}
  \item Collision resistance: $2^{128}$ security (quantum: $2^{85}$)
  \item Preimage resistance: $2^{256}$ security (quantum: $2^{128}$)
  \item Sufficient for all security requirements
\end{itemize}

\subsubsection{Signature Security}

\begin{table}[h]
\centering
\caption{Signature security levels}
\label{tab:sig-security}
\begin{tabular}{@{}lccc@{}}
\toprule
\textbf{Algorithm} & \textbf{Classical} & \textbf{Quantum} & \textbf{Use} \\
\midrule
CLSAG & 128-bit & 0 & Ring signatures \\
ML-DSA-65 & 192-bit & 128-bit & Minting \\
\bottomrule
\end{tabular}
\end{table}

\subsubsection{Key Encapsulation Security}

ML-KEM-768 provides:
\begin{itemize}
  \item IND-CCA2 security against quantum adversaries
  \item 192-bit classical security / 128-bit quantum security
  \item Based on hardness of MLWE problem
\end{itemize}

\subsection{Attack Resistance}

\subsubsection{Timing Attacks}

\textbf{Transaction timing}:
\begin{itemize}
  \item Dandelion++ randomizes propagation timing
  \item Batching obscures submission time
  \item Mempool privacy prevents timing correlation
\end{itemize}

\textbf{Decoy selection}:
\begin{itemize}
  \item Age distribution matches empirical spend patterns
  \item Randomization within constraints
  \item No deterministic selection
\end{itemize}

\subsubsection{Transaction Graph Analysis}

Several features resist graph analysis:

\begin{itemize}
  \item \textbf{Ring signatures}: True input hidden among 20 possibilities
  \item \textbf{Stealth addresses}: Each output has unique one-time key
  \item \textbf{Hidden amounts}: Cannot trace by amount matching
  \item \textbf{Multiple outputs}: Change output indistinguishable
\end{itemize}

\subsubsection{Sybil Attacks}

Cluster-based progressive fees resist Sybil attacks:

\begin{itemize}
  \item Splitting coins preserves cluster ancestry
  \item Creating new identities doesn't reduce fees
  \item Only genuine economic activity changes cluster factor
\end{itemize}

\subsubsection{Denial of Service}

\textbf{Network-level}:
\begin{itemize}
  \item Rate limiting per peer
  \item Reputation scoring
  \item Connection limits
  \item Resource bounds
\end{itemize}

\textbf{Consensus-level}:
\begin{itemize}
  \item PoW requires resources to propose blocks
  \item Invalid proposals rejected before propagation
  \item SCP messages bounded by quorum size
\end{itemize}

\textbf{Transaction-level}:
\begin{itemize}
  \item Minimum fee requirement
  \item Validation before relay
  \item Mempool size limits
\end{itemize}

\subsubsection{Eclipse Attacks}

\textbf{Mitigation}:
\begin{itemize}
  \item Diverse peer selection (multiple ASNs, countries)
  \item Outbound connection preference
  \item Bootstrap node diversity
  \item Detection of peer manipulation
\end{itemize}

\subsubsection{Selfish Mining}

\Botho's hybrid consensus changes selfish mining dynamics:

\begin{itemize}
  \item Block withholding delays finalization but doesn't create advantage
  \item SCP selects among available proposals, not necessarily first
  \item No ``longest chain'' to game
  \item Withholding risks losing the block entirely
\end{itemize}

\subsection{Formal Security Properties}

\subsubsection{Safety}

\begin{property}[Value Conservation]
For any valid block, the sum of all transaction outputs plus fees equals
the sum of all inputs plus block reward.
\end{property}

\begin{property}[No Inflation]
The total supply at height $h$ is exactly the sum of all block rewards
up to $h$, minus all burned fees.
\end{property}

\begin{property}[Key Image Uniqueness]
Each output can be spent exactly once; the key image uniquely identifies
the spent output.
\end{property}

\subsubsection{Liveness}

\begin{property}[Transaction Inclusion]
A valid transaction with sufficient fee will be included in a block within
bounded time, assuming network synchrony and honest quorum majority.
\end{property}

\begin{property}[Consensus Progress]
If the network is eventually synchronous and quorum intersection holds,
SCP will eventually externalize a value for each slot.
\end{property}

\subsection{Security Comparison}

\begin{table}[h]
\centering
\caption{Security comparison with other privacy coins}
\label{tab:security-comparison}
\begin{tabular}{@{}lcccc@{}}
\toprule
\textbf{Property} & \textbf{\Botho} & \textbf{Monero} & \textbf{Zcash} & \textbf{Grin} \\
\midrule
PQ recipient privacy & \checkmark & --- & --- & --- \\
Ring size & 20 & 16 & N/A & N/A \\
Amount hiding & IT-secure & IT-secure & IT-secure & IT-secure \\
Mandatory privacy & \checkmark & \checkmark & --- & \checkmark \\
Deterministic finality & \checkmark & --- & --- & --- \\
Trusted setup & --- & --- & \checkmark & --- \\
\bottomrule
\end{tabular}
\end{table}

(IT-secure = information-theoretically secure)

\subsection{Known Limitations}

\subsubsection{Classical Ring Signature Vulnerability}

A sufficiently powerful quantum computer could break CLSAG and identify
signers retroactively. Mitigation:
\begin{itemize}
  \item Ring signature data alone doesn't prove spending
  \item Economic value of retroactive deanonymization is limited
  \item Future protocol upgrade to post-quantum rings is possible
\end{itemize}

\subsubsection{Metadata Leakage}

Despite cryptographic privacy:
\begin{itemize}
  \item Transaction size reveals input/output count
  \item Timing correlation possible with powerful adversary
  \item Network-level deanonymization without Tor/I2P
\end{itemize}

\subsubsection{Implementation Risks}

\begin{itemize}
  \item Side-channel attacks on key material
  \item Random number generator quality
  \item Memory safety issues
  \item Timing side channels in cryptographic operations
\end{itemize}

All reference implementation cryptographic code uses constant-time
algorithms and is subject to ongoing security audit.

