% ============================================================================
% Section 2: Related Work
% ============================================================================

\section{Related Work}
\label{sec:related}

\subsection{Privacy Cryptocurrencies}

\subsubsection{CryptoNote and Monero}

The CryptoNote protocol~\cite{cryptonote} introduced two foundational
privacy techniques: \textit{stealth addresses} for recipient privacy and
\textit{ring signatures}~\cite{rivest2001} for sender privacy. Monero~\cite{monero}, the
most widely adopted CryptoNote implementation, has evolved significantly
since its 2014 launch.

Monero's current protocol uses:
\begin{itemize}
  \item \textbf{CLSAG ring signatures}~\cite{clsag}: Concise Linkable
    Spontaneous Anonymous Group signatures, providing approximately 45\%
    size reduction over the earlier MLSAG scheme~\cite{mlsag,mrl0005}.
  \item \textbf{Pedersen commitments}~\cite{pedersen}: Amount hiding with additive
    homomorphism for value conservation verification.
  \item \textbf{Bulletproofs}~\cite{bulletproofs}: Logarithmic-size range
    proofs ensuring committed amounts are non-negative.
  \item \textbf{ECDH stealth addresses}: One-time keys derived via
    elliptic curve Diffie-Hellman.
\end{itemize}

Monero achieves strong privacy but faces several limitations that \Botho
addresses:
\begin{enumerate}
  \item \textbf{No quantum resistance}: All cryptographic primitives are
    vulnerable to quantum attack.
  \item \textbf{Probabilistic finality}: Nakamoto consensus requires
    multiple confirmations; reorgs remain possible.
  \item \textbf{No anti-hoarding mechanism}: No economic incentive against
    wealth concentration.
\end{enumerate}

\subsubsection{Zcash}

Zcash~\cite{zcash} pioneered the use of zero-knowledge proofs
(zk-SNARKs)~\cite{zksnark} for cryptocurrency privacy. Unlike ring
signatures, which hide the sender among a set of decoys, zk-SNARKs
prove transaction validity without revealing any transaction details.

Zcash's privacy model differs fundamentally from \Botho:
\begin{itemize}
  \item \textbf{Opt-in privacy}: Most Zcash transactions are transparent;
    shielded transactions require explicit user action.
  \item \textbf{Trusted setup history}: The original Sprout ceremony
    required trust in participant integrity. Later upgrades (Sapling,
    Orchard) reduced but did not eliminate setup requirements.
  \item \textbf{Computational cost}: Proof generation requires significant
    computation, limiting usability on constrained devices.
\end{itemize}

\Botho chooses mandatory privacy with ring signatures---simpler, faster,
and requiring no trusted setup---accepting the tradeoff of probabilistic
rather than cryptographic anonymity.

\subsubsection{Zcash Orchard}

Zcash's Orchard protocol~\cite{orchard} represents the latest evolution of
zk-SNARK-based privacy, deployed in 2022. Key innovations include:

\begin{itemize}
  \item \textbf{Halo 2 proof system}: Eliminates trusted setup via recursive
    proof composition. Proofs verify in constant time regardless of circuit
    complexity.
  \item \textbf{Unified addresses}: Single address format supporting
    transparent, Sapling, and Orchard funds, simplifying user experience.
  \item \textbf{Pallas/Vesta curves}: Purpose-built curves enabling efficient
    recursive proofs without pairing-based cryptography.
  \item \textbf{Action-based model}: Transactions consist of ``actions'' that
    can spend and create notes simultaneously, improving privacy analysis.
\end{itemize}

Despite these advances, Orchard maintains characteristics \Botho avoids:
\begin{itemize}
  \item \textbf{Optional shielding}: Transparent transactions remain possible
    and common, fragmenting the anonymity set.
  \item \textbf{Heavy computation}: Proof generation requires 2--5 seconds on
    modern hardware, challenging for mobile devices.
  \item \textbf{Classical cryptography}: Pallas/Vesta curves remain
    vulnerable to Shor's algorithm.
\end{itemize}

\subsubsection{Firo (Lelantus Spark)}

Firo (formerly Zcoin) introduced Lelantus Spark~\cite{spark} in 2023,
combining Spark addresses with Lelantus-style one-out-of-many proofs:

\begin{itemize}
  \item \textbf{Spark addresses}: Diversified addresses with incoming view
    key separation, similar to Monero's subaddresses but with additional
    flexibility.
  \item \textbf{One-out-of-many proofs}: Alternative to ring signatures
    providing similar sender anonymity with different tradeoffs.
  \item \textbf{No trusted setup}: Like ring signatures, Lelantus requires
    no parameter ceremony.
\end{itemize}

Lelantus Spark offers comparable privacy to ring signatures with different
performance characteristics. \Botho's choice of CLSAG reflects:
\begin{itemize}
  \item Extensive security analysis and real-world deployment (Monero)
  \item Smaller proof sizes for typical transactions
  \item Better understood cryptographic assumptions
\end{itemize}

\subsubsection{Secret Network}

Secret Network~\cite{secretnetwork} takes a fundamentally different
approach: privacy through trusted execution environments (TEEs):

\begin{itemize}
  \item \textbf{Intel SGX enclaves}: Computation occurs in hardware-isolated
    regions; even node operators cannot observe transaction details.
  \item \textbf{Secret contracts}: Programmable privacy with Turing-complete
    smart contracts---a capability pure cryptographic approaches cannot
    efficiently achieve.
  \item \textbf{Fast execution}: No proof generation overhead; privacy adds
    minimal latency.
\end{itemize}

The TEE approach involves different security assumptions:
\begin{itemize}
  \item \textbf{Hardware trust}: Security depends on Intel's implementation
    and absence of hardware vulnerabilities.
  \item \textbf{Side-channel attacks}: SGX has faced multiple side-channel
    attacks (Spectre, Meltdown variants, etc.)~\cite{sgx-attacks}.
  \item \textbf{Centralized manufacturing}: Hardware availability depends
    on Intel's production and cooperation.
\end{itemize}

\Botho's purely cryptographic approach provides:
\begin{itemize}
  \item No hardware trust assumptions
  \item Mathematically verifiable security properties
  \item Resilience against physical side-channel attacks
\end{itemize}

\subsubsection{Aztec}

Aztec~\cite{aztec} brings privacy to Ethereum through a zk-rollup Layer 2:

\begin{itemize}
  \item \textbf{Private L2}: Transactions are private within Aztec; only
    validity proofs are posted to Ethereum.
  \item \textbf{Composability}: Private smart contracts can interact with
    Ethereum DeFi through controlled interfaces.
  \item \textbf{PLONK proofs}: Uses efficient universal zkSNARKs with
    updateable trusted setup.
\end{itemize}

Aztec's L2 approach offers interesting tradeoffs:
\begin{itemize}
  \item \textbf{Inherits L1 security}: Settlement finality comes from
    Ethereum's consensus.
  \item \textbf{Limited decentralization}: Sequencer sets remain somewhat
    centralized during early development.
  \item \textbf{Bridging friction}: Moving funds between L1 and L2 introduces
    latency and potential vulnerabilities.
\end{itemize}

\Botho operates as an independent L1, avoiding bridge security concerns while
sacrificing Ethereum composability.

\subsubsection{MimbleWimble}

MimbleWimble~\cite{mimblewimble} achieves privacy through transaction
cut-through and kernel aggregation. Implementations include Grin~\cite{grin} and Beam~\cite{beam}.

Key limitations addressed by \Botho:
\begin{itemize}
  \item \textbf{Interactive transactions}: MimbleWimble requires sender and
    recipient to exchange data, complicating user experience.
  \item \textbf{Linkability attacks}: Research has demonstrated significant
    transaction graph leakage~\cite{mw-attack}.
\end{itemize}

\subsection{Post-Quantum Approaches}

\subsubsection{Quantum Threat Model}

Shor's algorithm~\cite{shor} enables quantum computers to solve the discrete
logarithm and integer factorization problems in polynomial time, breaking
ECDSA, ECDH, and related schemes. Grover's algorithm~\cite{grover} provides
quadratic speedup for symmetric key search, effectively halving security
levels.

The ``harvest now, decrypt later'' threat is particularly relevant for
blockchain systems: adversaries may record encrypted/signed data today
for decryption when quantum computers become available. Since blockchain
data is permanent and public, this threat applies to all historical
transactions.

\subsubsection{Existing Post-Quantum Cryptocurrencies}

\textbf{QRL (Quantum Resistant Ledger)}~\cite{qrl} uses XMSS hash-based
signatures. While quantum-resistant, XMSS is stateful---requiring careful
key management to avoid catastrophic signature reuse.

\textbf{IOTA} has explored Winternitz one-time signatures, facing similar
statefulness challenges.

Research proposals for post-quantum ring signatures~\cite{pq-ring-sig,pqringsurvey}
exist but impose significant size overhead
($\sim$50$\times$ classical signatures).

\subsubsection{Post-Quantum Ring Signature Research}

The search for practical post-quantum ring signatures remains an active
research area. Current approaches include:

\textbf{Lattice-based constructions}~\cite{lattice-ring-sig}:
\begin{itemize}
  \item Based on Ring-LWE or Module-LWE problems
  \item Signature sizes: 15--50 KB for ring size 16
  \item Verification: $\sim$10--50 ms
  \item Primary challenge: Size scales poorly with ring size
\end{itemize}

\textbf{Hash-based ring signatures}:
\begin{itemize}
  \item Rely only on hash function security (conservative assumption)
  \item Even larger signatures than lattice approaches
  \item Better understood security reductions
\end{itemize}

\textbf{Isogeny-based constructions}:
\begin{itemize}
  \item Based on supersingular isogeny problems
  \item More compact than lattice approaches
  \item Slower computation; SIDH attacks~\cite{sidh-attack} have reduced
    confidence in isogeny hardness assumptions
\end{itemize}

\textbf{Comparison to \Botho's CLSAG}:
\begin{center}
\begin{tabular}{@{}lrrr@{}}
\toprule
\textbf{Scheme} & \textbf{Ring 16} & \textbf{Ring 32} & \textbf{Quantum Safe} \\
\midrule
CLSAG & 704 B & 1,376 B & No \\
Lattice Ring (typical) & 35 KB & 70 KB & Yes \\
Hash-based Ring & 100+ KB & 200+ KB & Yes \\
\bottomrule
\end{tabular}
\end{center}

\textbf{Timeline assessment}: Practical post-quantum ring signatures achieving
sub-10 KB sizes for ring size 16 remain 5--10 years away based on current
research trajectories. \Botho's hybrid architecture is designed to accommodate
migration when such constructions mature, while providing immediate quantum
resistance for permanent on-chain data (recipient identities).

\textbf{Migration path}: \Botho's modular cryptographic design allows future
protocol upgrades to replace CLSAG with post-quantum alternatives. The
signature abstraction layer (Section~\ref{sec:implementation}) isolates ring
signature specifics from transaction processing, enabling migration via soft
fork when practical PQ ring signatures emerge.

\subsubsection{Hybrid Approaches}

NIST's post-quantum cryptography standardization~\cite{nist-pqc} produced
ML-KEM (formerly Kyber)~\cite{fips203,kyber} for key encapsulation and ML-DSA (formerly
Dilithium)~\cite{fips204,dilithium} for signatures. These lattice-based schemes provide strong
post-quantum security with reasonable performance.

\Botho's hybrid approach applies these standards strategically:
\begin{itemize}
  \item \textbf{ML-KEM-768} for stealth addresses: Recipient identity is
    permanent on-chain data requiring permanent protection.
  \item \textbf{ML-DSA-65} for minting: Block producer identity is public
    and must be unforgeable long-term.
  \item \textbf{CLSAG} for ring signatures: Sender privacy is ephemeral;
    classical cryptography provides efficiency.
\end{itemize}

This selective application achieves meaningful quantum resistance while
maintaining practical transaction sizes.

\subsection{Consensus Mechanisms}

\subsubsection{Nakamoto Consensus}

Bitcoin's Nakamoto consensus~\cite{nakamoto2008bitcoin} achieves agreement
through proof-of-work and longest-chain selection. Its key properties:
\begin{itemize}
  \item \textbf{Permissionless}: Anyone can participate by providing
    hashpower.
  \item \textbf{Probabilistic finality}: Transactions become exponentially
    harder to reverse with additional confirmations, but reorgs remain
    theoretically possible.
  \item \textbf{Slow}: Bitcoin's 10-minute blocks and 6-confirmation
    standard yield $\sim$60-minute finality.
\end{itemize}

\subsubsection{Classical BFT}

Byzantine Fault Tolerant protocols like PBFT~\cite{pbft} achieve
deterministic finality with $3f+1$ nodes tolerating $f$ Byzantine failures~\cite{byzantine}.
However, classical BFT requires:
\begin{itemize}
  \item Known, fixed participant set
  \item $O(n^2)$ message complexity
  \item Synchrony assumptions for liveness
\end{itemize}

These requirements conflict with permissionless cryptocurrency design.

\subsubsection{Stellar Consensus Protocol}

SCP~\cite{scp} introduces \textit{federated Byzantine agreement}, achieving
BFT-like properties with open membership:
\begin{itemize}
  \item \textbf{Quorum slices}: Each node declares which nodes it trusts.
  \item \textbf{Quorum intersection}: Safety requires overlapping quorums.
  \item \textbf{Open membership}: New nodes can join by declaring trust
    relationships.
  \item \textbf{Fast finality}: Consensus typically completes in seconds.
\end{itemize}

\Botho combines PoW block proposal (for permissionless participation and
fair distribution) with SCP finalization (for fast deterministic finality).

\subsection{Economic Mechanisms}

\subsubsection{Fixed Supply}

Bitcoin's 21 million coin cap creates absolute scarcity but raises
long-term security concerns. As block rewards diminish, network security
depends increasingly on transaction fees. If fee revenue proves
insufficient, security may degrade.

\subsubsection{Tail Emission}

Monero's perpetual tail emission ($\sim$0.6 XMR per block indefinitely)
ensures permanent miner incentive. This trades absolute scarcity for
security sustainability---a tradeoff \Botho embraces.

\subsubsection{Demurrage}

Demurrage currencies~\cite{gesell} (Freigeld, Chiemgauer) impose holding costs through
time-based value decay. While economically interesting, cryptographic
implementation is challenging and user experience suffers.

\subsubsection{Novel: Progressive Fees}

\Botho introduces \textit{cluster-based progressive fees}---a novel
mechanism that:
\begin{itemize}
  \item Increases transaction costs for concentrated wealth
  \item Tracks coin \textit{ancestry} rather than current ownership
  \item Resists Sybil attacks (splitting coins doesn't reduce fees)
  \item Preserves privacy (no identity required)
\end{itemize}

Combined with fee burning, this creates economic pressure toward
circulation without explicit demurrage.
