% ============================================================================
% Section 2: Related Work
% ============================================================================

\section{Related Work}
\label{sec:related}

\subsection{Privacy Cryptocurrencies}

\subsubsection{CryptoNote and Monero}

The CryptoNote protocol introduced two foundational
privacy techniques: \textit{stealth addresses} for recipient privacy and
\textit{ring signatures} for sender privacy. Monero, the
most widely adopted CryptoNote implementation, has evolved significantly
since its 2014 launch.

Monero's current protocol uses:
\begin{itemize}
  \item \textbf{CLSAG ring signatures}: Concise Linkable
    Spontaneous Anonymous Group signatures, providing approximately 45\%
    size reduction over the earlier MLSAG scheme.
  \item \textbf{Pedersen commitments}: Amount hiding with additive
    homomorphism for value conservation verification.
  \item \textbf{Bulletproofs}: Logarithmic-size range
    proofs ensuring committed amounts are non-negative.
  \item \textbf{ECDH stealth addresses}: One-time keys derived via
    elliptic curve Diffie-Hellman.
\end{itemize}

Monero achieves strong privacy but faces several limitations that \Botho
addresses:
\begin{enumerate}
  \item \textbf{No quantum resistance}: All cryptographic primitives are
    vulnerable to quantum attack.
  \item \textbf{Probabilistic finality}: Nakamoto consensus requires
    multiple confirmations; reorgs remain possible.
  \item \textbf{No anti-hoarding mechanism}: No economic incentive against
    wealth concentration.
\end{enumerate}

\subsubsection{Zcash}

Zcash pioneered the use of zero-knowledge proofs
(zk-SNARKs) for cryptocurrency privacy. Unlike ring
signatures, which hide the sender among a set of decoys, zk-SNARKs
prove transaction validity without revealing any transaction details.

Zcash's privacy model differs fundamentally from \Botho:
\begin{itemize}
  \item \textbf{Opt-in privacy}: Most Zcash transactions are transparent;
    shielded transactions require explicit user action.
  \item \textbf{Trusted setup history}: The original Sprout ceremony
    required trust in participant integrity. Later upgrades (Sapling,
    Orchard) reduced but did not eliminate setup requirements.
  \item \textbf{Computational cost}: Proof generation requires significant
    computation, limiting usability on constrained devices.
\end{itemize}

\Botho chooses mandatory privacy with ring signatures---simpler, faster,
and requiring no trusted setup---accepting the tradeoff of probabilistic
rather than cryptographic anonymity.

\subsubsection{MimbleWimble}

MimbleWimble achieves privacy through transaction
cut-through and kernel aggregation. Implementations include Grin and Beam.

Key limitations addressed by \Botho:
\begin{itemize}
  \item \textbf{Interactive transactions}: MimbleWimble requires sender and
    recipient to exchange data, complicating user experience.
  \item \textbf{Linkability attacks}: Research has demonstrated significant
    transaction graph leakage.
\end{itemize}

\subsection{Post-Quantum Approaches}

\subsubsection{Quantum Threat Model}

Shor's algorithm enables quantum computers to solve the discrete
logarithm and integer factorization problems in polynomial time, breaking
ECDSA, ECDH, and related schemes. Grover's algorithm provides
quadratic speedup for symmetric key search, effectively halving security
levels.

The ``harvest now, decrypt later'' threat is particularly relevant for
blockchain systems: adversaries may record encrypted/signed data today
for decryption when quantum computers become available. Since blockchain
data is permanent and public, this threat applies to all historical
transactions.

\subsubsection{Existing Post-Quantum Cryptocurrencies}

\textbf{QRL (Quantum Resistant Ledger)} uses XMSS hash-based
signatures. While quantum-resistant, XMSS is stateful---requiring careful
key management to avoid catastrophic signature reuse.

\textbf{IOTA} has explored Winternitz one-time signatures, facing similar
statefulness challenges.

Research proposals for post-quantum ring signatures
exist but impose significant size overhead
($\sim$50$\times$ classical signatures).

\subsubsection{Hybrid Approaches}

NIST's post-quantum cryptography standardization produced
ML-KEM (formerly Kyber) for key encapsulation and ML-DSA (formerly
Dilithium) for signatures. These lattice-based schemes provide strong
post-quantum security with reasonable performance.

\Botho's hybrid approach applies these standards strategically:
\begin{itemize}
  \item \textbf{ML-KEM-768} for stealth addresses: Recipient identity is
    permanent on-chain data requiring permanent protection.
  \item \textbf{ML-DSA-65} for minting: Block producer identity is public
    and must be unforgeable long-term.
  \item \textbf{CLSAG} for ring signatures: Sender privacy is ephemeral;
    classical cryptography provides efficiency.
\end{itemize}

This selective application achieves meaningful quantum resistance while
maintaining practical transaction sizes.

\subsection{Consensus Mechanisms}

\subsubsection{Nakamoto Consensus}

Bitcoin's Nakamoto consensus achieves agreement
through proof-of-work and longest-chain selection. Its key properties:
\begin{itemize}
  \item \textbf{Permissionless}: Anyone can participate by providing
    hashpower.
  \item \textbf{Probabilistic finality}: Transactions become exponentially
    harder to reverse with additional confirmations, but reorgs remain
    theoretically possible.
  \item \textbf{Slow}: Bitcoin's 10-minute blocks and 6-confirmation
    standard yield $\sim$60-minute finality.
\end{itemize}

\subsubsection{Classical BFT}

Byzantine Fault Tolerant protocols like PBFT achieve
deterministic finality with $3f+1$ nodes tolerating $f$ Byzantine failures.
However, classical BFT requires:
\begin{itemize}
  \item Known, fixed participant set
  \item $O(n^2)$ message complexity
  \item Synchrony assumptions for liveness
\end{itemize}

These requirements conflict with permissionless cryptocurrency design.

\subsubsection{Stellar Consensus Protocol}

SCP introduces \textit{federated Byzantine agreement}, achieving
BFT-like properties with open membership:
\begin{itemize}
  \item \textbf{Quorum slices}: Each node declares which nodes it trusts.
  \item \textbf{Quorum intersection}: Safety requires overlapping quorums.
  \item \textbf{Open membership}: New nodes can join by declaring trust
    relationships.
  \item \textbf{Fast finality}: Consensus typically completes in seconds.
\end{itemize}

\Botho combines PoW block proposal (for permissionless participation and
fair distribution) with SCP finalization (for fast deterministic finality).

\subsection{Economic Mechanisms}

\subsubsection{Fixed Supply}

Bitcoin's 21 million coin cap creates absolute scarcity but raises
long-term security concerns. As block rewards diminish, network security
depends increasingly on transaction fees. If fee revenue proves
insufficient, security may degrade.

\subsubsection{Tail Emission}

Monero's perpetual tail emission ($\sim$0.6 XMR per block indefinitely)
ensures permanent miner incentive. This trades absolute scarcity for
security sustainability---a tradeoff \Botho embraces.

\subsubsection{Demurrage}

Demurrage currencies (Freigeld, Chiemgauer) impose holding costs through
time-based value decay. While economically interesting, cryptographic
implementation is challenging and user experience suffers.

\subsubsection{Novel: Progressive Fees}

\Botho introduces \textit{cluster-based progressive fees}---a novel
mechanism that:
\begin{itemize}
  \item Increases transaction costs for concentrated wealth
  \item Tracks coin \textit{ancestry} rather than current ownership
  \item Resists Sybil attacks (splitting coins doesn't reduce fees)
  \item Preserves privacy (no identity required)
\end{itemize}

Combined with fee burning, this creates economic pressure toward
circulation without explicit demurrage.
