% ============================================================================
% Appendix E: Security Audit Guide
% ============================================================================

\section{Security Audit Guide}
\label{app:audit}

This appendix provides guidance for security auditors reviewing the \Botho
protocol and implementation.

\subsection{Security Claims}

The following security properties are claimed by \Botho. Auditors should
verify each claim against the implementation.

\subsubsection{Cryptographic Security}

\begin{table}[h]
\centering
\caption{Cryptographic security claims}
\label{tab:crypto-claims}
\begin{tabular}{@{}p{4cm}p{3cm}p{6cm}@{}}
\toprule
\textbf{Property} & \textbf{Primitive} & \textbf{Assumption} \\
\midrule
Recipient unlinkability & ML-KEM-768 & IND-CCA2 security of ML-KEM \\
Sender anonymity & CLSAG & DLP hardness in Ristretto255 \\
Amount confidentiality & Pedersen + BP & DLP hardness, random oracle model \\
Double-spend prevention & Key images & Collision resistance of $H_p$ \\
Minting authenticity & ML-DSA-65 & EUF-CMA security of ML-DSA \\
\bottomrule
\end{tabular}
\end{table}

\subsubsection{Consensus Security}

\begin{enumerate}
  \item \textbf{Fork freedom}: No two honest nodes externalize different
    blocks at the same height (Theorem~\ref{thm:fork-freedom})
  \item \textbf{Liveness}: The network eventually makes progress under
    partial synchrony
  \item \textbf{Byzantine tolerance}: Safety holds with up to $f$ Byzantine
    nodes per quorum slice
\end{enumerate}

\subsubsection{Economic Security}

\begin{enumerate}
  \item \textbf{Sybil resistance}: Splitting wealth does not reduce total fees
    (Theorem~\ref{thm:sybil-resistance})
  \item \textbf{Lottery fairness}: UTXO selection is uniformly random and
    unpredictable
  \item \textbf{Inflation bounds}: Supply growth follows the emission schedule
    exactly
\end{enumerate}

\subsection{Threat Model}

Auditors should consider the following adversary capabilities:

\begin{table}[h]
\centering
\caption{Threat model assumptions}
\label{tab:threat-model}
\begin{tabular}{@{}lp{9cm}@{}}
\toprule
\textbf{Adversary Type} & \textbf{Capabilities} \\
\midrule
Network adversary & Can observe, delay, or reorder messages; cannot forge
signatures or break encryption \\
Computational adversary & Polynomial-time bounded; cannot solve DLP or
break ML-KEM \\
Byzantine nodes & Up to $f$ nodes per quorum slice may behave arbitrarily \\
Quantum adversary & Future: can break DLP but not lattice problems \\
\bottomrule
\end{tabular}
\end{table}

\subsection{Audit Scope}

\subsubsection{Critical Components}

The following components require thorough review:

\begin{enumerate}
  \item \textbf{Key derivation} (\texttt{crypto/keys.rs})
    \begin{itemize}
      \item BIP39 mnemonic handling
      \item SLIP-10 derivation paths
      \item ML-KEM key generation from seed
    \end{itemize}

  \item \textbf{Stealth addresses} (\texttt{crypto/stealth.rs})
    \begin{itemize}
      \item ML-KEM encapsulation/decapsulation
      \item One-time key derivation
      \item Output scanning correctness
    \end{itemize}

  \item \textbf{Ring signatures} (\texttt{crypto/clsag.rs})
    \begin{itemize}
      \item Signature generation and verification
      \item Key image computation
      \item Ring member selection
    \end{itemize}

  \item \textbf{Confidential transactions} (\texttt{crypto/bulletproofs.rs})
    \begin{itemize}
      \item Pedersen commitment arithmetic
      \item Range proof generation and verification
      \item Value conservation checks
    \end{itemize}

  \item \textbf{Consensus} (\texttt{consensus/scp.rs})
    \begin{itemize}
      \item Quorum slice validation
      \item Ballot protocol state machine
      \item Externalization logic
    \end{itemize}

  \item \textbf{Cluster tags} (\texttt{economics/cluster.rs})
    \begin{itemize}
      \item Tag inheritance calculation
      \item Fee multiplier computation
      \item Decay mechanism
    \end{itemize}
\end{enumerate}

\subsubsection{Test Vectors}

Auditors should verify the following test vectors:

\begin{lstlisting}[caption={Key derivation test vector}]
Mnemonic: "abandon abandon ... about" (standard 24-word)
Expected view key (hex): 0x7b3a...
Expected spend key (hex): 0x4c1f...
Expected ML-KEM public key (first 32 bytes): 0x9d2e...
\end{lstlisting}

\begin{lstlisting}[caption={Ring signature test vector}]
Message: 0x0000...0000 (32 zero bytes)
Ring size: 20
Real index: 7
Expected key image (hex): 0x5f8c...
Signature must verify: true
\end{lstlisting}

\begin{lstlisting}[caption={Cluster factor test vector}]
Cluster wealth: 1,000,000 BTH
Steepness parameter: 100,000
Expected factor: 5.5x
\end{lstlisting}

\subsection{Known Limitations}

The following are acknowledged limitations, not vulnerabilities:

\begin{enumerate}
  \item \textbf{Classical ring signatures}: CLSAG provides sender privacy
    against classical adversaries only. A future quantum computer could
    potentially identify signers retroactively.

  \item \textbf{Metadata leakage}: Transaction size reveals approximate
    input/output count. Timing analysis may leak information despite
    Dandelion++.

  \item \textbf{Ring selection bias}: The decoy selection algorithm uses
    a gamma distribution that may not perfectly match real spend patterns.

  \item \textbf{Trusted setup}: Bulletproofs require no trusted setup, but
    the choice of generators $G$ and $H$ must be verifiably random.
\end{enumerate}

\subsection{Reporting}

Security issues should be reported to:

\begin{itemize}
  \item \textbf{Email}: security@botho.org (PGP key on website)
  \item \textbf{Severity levels}: Critical, High, Medium, Low, Informational
  \item \textbf{Bug bounty}: Available for critical and high severity issues
\end{itemize}
