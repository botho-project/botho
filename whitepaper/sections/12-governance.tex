% ============================================================================
% Section 12: Governance and Protocol Upgrades
% ============================================================================

\section{Governance and Protocol Upgrades}
\label{sec:governance}

Protocol evolution is essential for long-term viability. \Botho employs
a structured governance process that balances agility with stability,
enabling necessary upgrades while protecting users from disruptive changes.

\subsection{Design Principles}

\subsubsection{Conservative by Default}

The protocol favors stability over rapid iteration:

\begin{itemize}
  \item Changes require demonstrated necessity, not merely novelty
  \item Backward compatibility is preserved wherever feasible
  \item Breaking changes require extended notice and migration support
  \item Security fixes take precedence over feature additions
\end{itemize}

\subsubsection{Transparency and Predictability}

All governance processes are public and follow documented procedures:

\begin{itemize}
  \item Proposal discussions occur in public forums
  \item Technical specifications are peer-reviewed before implementation
  \item Activation timelines are announced well in advance
  \item Post-activation monitoring reports are published
\end{itemize}

\subsection{Protocol Versioning}

\subsubsection{Version Scheme}

\Botho uses semantic versioning with blockchain-specific semantics:

\begin{lstlisting}[caption={Version number interpretation}]
MAJOR.MINOR.PATCH

MAJOR: Consensus-breaking changes (hard fork required)
MINOR: New features, backward compatible
PATCH: Bug fixes, no new features
\end{lstlisting}

\textbf{Examples}:
\begin{itemize}
  \item $1.0.0 \to 1.0.1$: Security patch, no action required
  \item $1.0.0 \to 1.1.0$: New transaction type added, old nodes still valid
  \item $1.0.0 \to 2.0.0$: Consensus rule change, upgrade mandatory
\end{itemize}

\subsubsection{Version Negotiation}

Nodes advertise their protocol version during handshake:

\begin{lstlisting}[caption={Version handshake}]
VersionMessage {
    protocol_version: (u16, u16, u16),  // MAJOR.MINOR.PATCH
    min_supported: (u16, u16, u16),      // Minimum peer version
    features: BitFlags,                   // Optional feature support
    user_agent: String,                   // Implementation identifier
}
\end{lstlisting}

Nodes with incompatible versions (different MAJOR) disconnect gracefully
with an informative error message.

\subsection{Upgrade Mechanisms}

\subsubsection{Soft Forks}

Soft forks tighten consensus rules---old nodes accept new blocks, but new
nodes may reject some blocks old nodes would accept.

\textbf{Use cases}:
\begin{itemize}
  \item Adding new transaction validation rules
  \item Restricting previously-allowed behaviors
  \item Deprecating obsolete features
\end{itemize}

\textbf{Activation process}:
\begin{enumerate}
  \item \textbf{Proposal}: BIP-style document published with technical specification
  \item \textbf{Implementation}: Reference client updated with activation logic
  \item \textbf{Signaling}: Miners include version bits in block headers
  \item \textbf{Lock-in}: When 95\% of blocks in a 2016-block window signal support
  \item \textbf{Activation}: Rules enforced after additional 2016-block grace period
\end{enumerate}

\begin{lstlisting}[caption={Soft fork signaling in block header}]
BlockHeader {
    version: u32,  // Bits 0-28: version, bits 29-31: soft fork signals
    // ...
}

// Check signaling for proposal #3
fn signals_proposal(header: &BlockHeader, proposal: u8) -> bool {
    (header.version >> (29 + proposal)) & 1 == 1
}
\end{lstlisting}

\subsubsection{Hard Forks}

Hard forks relax or modify consensus rules---old nodes reject new blocks.

\textbf{Use cases}:
\begin{itemize}
  \item Fundamental protocol changes (new cryptographic primitives)
  \item Emergency security responses
  \item Removal of technical debt
\end{itemize}

\textbf{Activation process}:
\begin{enumerate}
  \item \textbf{Proposal}: Extended discussion period (minimum 6 months)
  \item \textbf{Specification}: Formal specification with test vectors
  \item \textbf{Implementation}: All major clients implement changes
  \item \textbf{Testnet}: Minimum 3-month testnet validation
  \item \textbf{Announcement}: Activation height announced 3+ months ahead
  \item \textbf{Activation}: At specified block height
\end{enumerate}

\textbf{Mandatory upgrade policy}: Users have minimum 6 months warning
before hard fork activation, except for emergency security fixes.

\subsection{Proposal Process}

\subsubsection{Botho Improvement Proposals (BIPs)}

Formal proposals follow a structured format:

\begin{lstlisting}[caption={BIP document structure}]
BIP-XXXX: Title

Status: Draft | Review | Accepted | Final | Rejected | Withdrawn
Type: Standards Track | Informational | Process
Layer: Consensus | Network | Application

Abstract: [200 words max]

Motivation: Why is this change needed?

Specification: Technical details

Rationale: Design decisions and alternatives considered

Backward Compatibility: Impact analysis

Test Cases: Validation criteria

Reference Implementation: Link to code

Security Considerations: Threat analysis
\end{lstlisting}

\subsubsection{Proposal Lifecycle}

\begin{enumerate}
  \item \textbf{Draft}: Author publishes initial proposal for feedback
  \item \textbf{Review}: Community discussion, technical review
  \item \textbf{Accepted}: Core maintainers approve for implementation
  \item \textbf{Final}: Implemented and deployed (or rejected/withdrawn)
\end{enumerate}

\textbf{Review criteria}:
\begin{itemize}
  \item Technical soundness and security analysis
  \item Backward compatibility assessment
  \item Implementation complexity vs. benefit
  \item Community consensus on necessity
\end{itemize}

\subsection{Emergency Procedures}

\subsubsection{Security Vulnerabilities}

Critical vulnerabilities follow expedited handling:

\begin{enumerate}
  \item \textbf{Discovery}: Reported via secure channel (security@botho.io)
  \item \textbf{Assessment}: Core team evaluates severity within 24 hours
  \item \textbf{Patch development}: Fix developed in private
  \item \textbf{Coordinated disclosure}: Major node operators notified
  \item \textbf{Public release}: Patch released with advisory
  \item \textbf{Post-mortem}: Public analysis after fix deployed
\end{enumerate}

\textbf{Severity classification}:
\begin{table}[h]
\centering
\caption{Vulnerability severity levels}
\label{tab:severity}
\begin{tabular}{@{}lll@{}}
\toprule
\textbf{Level} & \textbf{Definition} & \textbf{Response Time} \\
\midrule
Critical & Funds at immediate risk & 24--48 hours \\
High & Exploitable with effort & 1--2 weeks \\
Medium & Limited impact & 1--2 months \\
Low & Minimal risk & Next release \\
\bottomrule
\end{tabular}
\end{table}

\subsubsection{Network Emergencies}

Catastrophic events (sustained 51\% attack, consensus failure) trigger
emergency response:

\begin{enumerate}
  \item Core team assesses situation and coordinates response
  \item Emergency communication via all channels
  \item If necessary: checkpoint enforcement, emergency hard fork
  \item Post-incident analysis and preventive measures
\end{enumerate}

\textbf{Checkpoint authority}: In extreme emergencies, core developers may
publish signed checkpoints that compliant nodes will refuse to reorg past.
This is a last-resort mechanism that trades decentralization for safety.

\subsection{Deprecation Policy}

\subsubsection{Feature Deprecation}

Obsolete features are phased out gradually:

\begin{enumerate}
  \item \textbf{Deprecation notice}: Feature marked deprecated in release notes
  \item \textbf{Warning period}: Nodes log warnings when deprecated features used
  \item \textbf{Soft removal}: Feature disabled by default, available via flag
  \item \textbf{Hard removal}: Feature removed entirely (requires hard fork)
\end{enumerate}

\textbf{Minimum timeline}: 12 months from deprecation notice to hard removal.

\subsubsection{API Stability}

External interfaces (RPC, wallet format) follow stability guarantees:

\begin{itemize}
  \item \textbf{Stable APIs}: 24+ months support after deprecation
  \item \textbf{Experimental APIs}: May change without notice (clearly marked)
  \item \textbf{Internal APIs}: No stability guarantee
\end{itemize}

\subsection{Decentralization Considerations}

\subsubsection{Avoiding Governance Capture}

Governance mechanisms are designed to resist capture:

\begin{itemize}
  \item \textbf{No formal voting}: Rough consensus, not vote counting
  \item \textbf{Multiple implementations}: No single codebase controls protocol
  \item \textbf{Economic alignment}: Core developers hold BTH, aligned with users
  \item \textbf{Transparent process}: All decisions documented publicly
\end{itemize}

\subsubsection{Fork Rights}

Users retain the ultimate governance power---the right to fork:

\begin{itemize}
  \item Protocol specification is open and unencumbered
  \item Reference implementation is permissively licensed (MIT)
  \item No trademark restrictions on protocol-compatible implementations
  \item Community can reject controversial changes by running alternatives
\end{itemize}

\textbf{Philosophy}: Formal governance exists to coordinate, not to rule.
The community's ability to fork ensures that governance serves users rather
than capturing them.

\subsection{Upgrade Roadmap}

\subsubsection{Planned Future Upgrades}

\begin{table}[h]
\centering
\caption{Tentative upgrade roadmap}
\label{tab:roadmap}
\begin{tabular}{@{}lll@{}}
\toprule
\textbf{Feature} & \textbf{Type} & \textbf{Status} \\
\midrule
View key audit mode & Soft fork & Under review \\
Bulletproofs+ & Soft fork & Specification draft \\
PQ ring signatures & Hard fork & Research phase \\
Payment channels & Application & Design phase \\
\bottomrule
\end{tabular}
\end{table}

\textbf{Note}: Roadmap items are aspirational and subject to community
consensus. No timeline commitments are made for research-phase items.

