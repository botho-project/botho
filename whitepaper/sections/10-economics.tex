% ============================================================================
% Section 10: Economic Analysis
% ============================================================================

\section{Economic Analysis}
\label{sec:economics}

This section analyzes the game-theoretic properties of \Botho's economic
mechanisms and their long-term sustainability.

\subsection{Incentive Alignment}

\subsubsection{Miner Incentives}

Miners (block proposers) are incentivized to:

\begin{enumerate}
  \item \textbf{Include transactions}: Larger blocks have no disadvantage
    (fees are burned, not captured)
  \item \textbf{Maintain network health}: Block rewards depend on network
    value, not transaction ordering
  \item \textbf{Stay honest}: Invalid blocks are rejected by SCP
\end{enumerate}

\textbf{Why fee burning works}:

Without fee capture, miners have no incentive to:
\begin{itemize}
  \item Reorder transactions (MEV elimination)
  \item Censor low-fee transactions beyond minimum
  \item Create artificial congestion
\end{itemize}

\subsubsection{User Incentives}

Users are incentivized to:

\begin{enumerate}
  \item \textbf{Circulate rather than hoard}: Progressive fees and tail
    emission create holding costs
  \item \textbf{Maintain privacy}: Default privacy eliminates coordination
    problems
  \item \textbf{Run nodes}: Reduced resource requirements enable
    participation
\end{enumerate}

\subsubsection{Validator Incentives}

SCP validators (consensus participants) are incentivized by:

\begin{enumerate}
  \item \textbf{Reputation}: Being in quorum slices requires trustworthiness
  \item \textbf{Self-interest}: Validators typically also hold currency
  \item \textbf{Community standing}: Non-economic incentives matter for
    infrastructure nodes
\end{enumerate}

\subsection{Progressive Fee Analysis}

\subsubsection{Fee Distribution}

Under the cluster-based progressive fee system:

\begin{equation}
E[\text{fee}] = f_{\text{base}} \times \text{size} \times E[\text{cluster\_factor}]
\end{equation}

The expected cluster factor depends on wealth distribution:

\begin{table}[h]
\centering
\caption{Expected fees by wealth percentile}
\label{tab:fee-percentile}
\begin{tabular}{@{}lcc@{}}
\toprule
\textbf{Percentile} & \textbf{Factor} & \textbf{4KB Tx Fee} \\
\midrule
0--50 & 1.0--1.5 & 4--6 $\mu$BTH \\
50--90 & 1.5--3.0 & 6--12 $\mu$BTH \\
90--99 & 3.0--5.0 & 12--20 $\mu$BTH \\
99--100 & 5.0--6.0 & 20--24 $\mu$BTH \\
\bottomrule
\end{tabular}
\end{table}

\subsubsection{Redistribution Effect}

The lottery mechanism creates direct wealth transfer from high-activity,
high-cluster users to random UTXO holders:

\begin{equation}
\text{redistribution}_{\text{direct}} = 0.8 \times \sum_{\text{tx}} f_{\text{total}}
\end{equation}

Additionally, fee burning (20\%) redistributes indirectly to all holders
by reducing supply:

\begin{equation}
\text{redistribution}_{\text{indirect}} = 0.2 \times \sum_{\text{tx}} f_{\text{total}}
\end{equation}

\textbf{Progressive properties of lottery selection}:
\begin{itemize}
  \item Random UTXO selection statistically favors many small holders over
    few large holders
  \item Self-custody users (many UTXOs) receive more than custodial users
    (few UTXOs holding many users' funds)
  \item Net effect: redistribution from exchanges to individuals
\end{itemize}

\subsubsection{Anti-Hoarding Dynamics}

The combination of:
\begin{itemize}
  \item Tail emission ($\sim$2\% annual supply increase)
  \item Progressive fees (wealth-based costs)
  \item Lottery redistribution (80\% of fees to random UTXOs)
  \item Fee burning (20\% deflationary pressure)
\end{itemize}

Creates net pressure toward circulation:

\begin{equation}
\text{holding\_cost} = \text{dilution} - \text{lottery\_income} + \text{opportunity\_cost}
\end{equation}

Users with more UTXOs (typically from active participation) receive more
lottery income, partially offsetting dilution. Large holders with few UTXOs
experience full dilution with minimal lottery income.

\subsection{Game-Theoretic Analysis}

\subsubsection{Miner Strategy}

\textbf{Proposition}: Honest mining is a Nash equilibrium.

\begin{proof}[Proof sketch]
Consider miner $M$ with hashpower fraction $\alpha$:
\begin{itemize}
  \item Honest strategy: Expected reward $\alpha \times R$ per block
  \item Withholding: Risk of block rejection by SCP (no benefit)
  \item Invalid blocks: Rejected, wasted computation
  \item Transaction censorship: No benefit (fees burned anyway)
\end{itemize}
No deviation improves expected payoff.
\end{proof}

\subsubsection{Validator Strategy}

\textbf{Proposition}: Honest validation is incentive-compatible for
economically invested validators.

\begin{proof}[Proof sketch]
Validators typically hold currency. Byzantine behavior risks:
\begin{itemize}
  \item Network failure (total loss of holdings)
  \item Exclusion from quorum slices (loss of influence)
  \item Reputation damage (non-economic but real)
\end{itemize}
The expected loss exceeds any potential gain from misbehavior.
\end{proof}

\subsubsection{User Strategy}

\textbf{Proposition}: Using the system as designed is individually rational.

Users face choices:
\begin{itemize}
  \item \textbf{Privacy}: Mandatory, so no choice to make
  \item \textbf{Fee payment}: Required for transaction inclusion
  \item \textbf{Holding vs. spending}: Mild pressure toward spending,
    but not coercive
\end{itemize}

\subsection{Long-Term Sustainability}

\subsubsection{Security Budget}

The annual security budget (miner revenue) is:

\begin{equation}
\text{budget} = R_{\text{tail}} \times \text{blocks\_per\_year} \times \text{BTH\_price}
\end{equation}

With tail emission, this is guaranteed to be positive regardless of
transaction volume.

\subsubsection{Network Effect}

Privacy creates positive network effects:

\begin{itemize}
  \item More users $\rightarrow$ larger anonymity sets
  \item Larger anonymity sets $\rightarrow$ better privacy
  \item Better privacy $\rightarrow$ more users
\end{itemize}

This creates a virtuous cycle encouraging adoption.

\subsubsection{Fee Market Stability}

Fee burning eliminates fee market volatility:

\begin{itemize}
  \item No bidding wars for block inclusion
  \item Predictable costs for users
  \item No miner incentive to manipulate fees
\end{itemize}

The minimum fee provides sufficient spam resistance while remaining
affordable.

\subsection{Comparison with Alternatives}

\subsubsection{Bitcoin Model}

\begin{table}[h]
\centering
\caption{Bitcoin vs Botho economic comparison}
\label{tab:btc-comparison}
\begin{tabular}{@{}lcc@{}}
\toprule
\textbf{Property} & \textbf{Bitcoin} & \textbf{\Botho} \\
\midrule
Supply cap & 21M & None (tail) \\
Long-term inflation & 0\% & $\sim$2\% \\
Security funding & Fees only & Emission + burn \\
Fee predictability & Low & High \\
Wealth concentration & High & Moderate \\
\bottomrule
\end{tabular}
\end{table}

\subsubsection{Monero Model}

\begin{table}[h]
\centering
\caption{Monero vs Botho economic comparison}
\label{tab:xmr-comparison}
\begin{tabular}{@{}lcc@{}}
\toprule
\textbf{Property} & \textbf{Monero} & \textbf{\Botho} \\
\midrule
Tail emission & Yes & Yes \\
Fee destination & Miners & Burned \\
Progressive fees & No & Yes \\
Dynamic timing & No & Yes \\
\bottomrule
\end{tabular}
\end{table}

\subsection{Economic Risks}

\subsubsection{Low Adoption}

If adoption remains low:
\begin{itemize}
  \item Tail emission maintains miner incentive
  \item Progressive fees have minimal impact (few large holders)
  \item Network can persist indefinitely at low scale
\end{itemize}

\subsubsection{Mining Centralization}

Risks and mitigations:
\begin{itemize}
  \item \textbf{Risk}: ASIC development concentrates mining
  \item \textbf{Mitigation}: CPU-friendly PoW (RandomX-based)
  \item \textbf{Risk}: Pool concentration
  \item \textbf{Mitigation}: Solo mining viable with tail emission
\end{itemize}

\subsubsection{Validator Cartel}

If validators collude:
\begin{itemize}
  \item SCP's tiered structure provides defense
  \item Community can adjust quorum slices
  \item Economic self-interest opposes cartels
\end{itemize}

\subsection{Economic Constants Summary}

\begin{table}[h]
\centering
\caption{Economic parameters}
\label{tab:econ-params}
\begin{tabular}{@{}lrl@{}}
\toprule
\textbf{Parameter} & \textbf{Value} & \textbf{Rationale} \\
\midrule
Tail inflation & $\sim$2\% & Security sustainability \\
Max fee multiplier & 6$\times$ & Balance incentive/burden \\
Ring size & 20 & Privacy/size tradeoff \\
Min block time & 5s & Throughput ceiling \\
Max block time & 40s & Emission floor \\
\bottomrule
\end{tabular}
\end{table}

